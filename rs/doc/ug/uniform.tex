\subsection{Uniform mesh}
\label{sec:uniform-mesh}

The simplest mesh available, is the structured, uniform mesh. This is
specified by the keyword \texttt{UniformMesh}.
%
\begin{verbatim}
begin Mesh

  type UniformMesh

  ...

end
\end{verbatim}
%

In the following we describe the type--specific subsections of the
uniform mesh.

\subsubsection{Dimension}
\label{sec:mesh-dimension}

The dimension is specified with the keyword \texttt{dimension}
followed by either \texttt{1}, \texttt{2} or \texttt{3} representing a
1D, 2D or 3D mesh respectively:
%
\begin{verbatim}
begin Mesh

  ...

  dimension 3

  ...

end
\end{verbatim}
%
If the mesh is 2D, the domain is part of the $xy$-plane (before any
transformations). If it is 1D, it extends in the $x$-direction. If the
dimension is other than 3, the size-keywords are affected, see
Section~\ref{sec:sizes} below.

\subsubsection{Origin}
\label{sec:origin}

The mesh origin is given by the three keywords \texttt{X0},
\texttt{Y0} and \texttt{Z0}. Each keyword is followed by the
coordinate value of the respective direction:
%
\begin{verbatim}
begin Mesh

  ...

  X0  0.0
  Y0  1.0
  Z0  0.0

  ...

end
\end{verbatim}


\subsubsection{Sizes}
\label{sec:sizes}

The actual mesh geometry is given by the number of elements and the
element sizes in each direction. The keywords \texttt{Nx}, \texttt{Ny}
and \texttt{Nz} specify the number of elements, while the sizes are
defined by the keywords \texttt{Dx}, \texttt{Dy} and \texttt{Dz}:
%
\begin{verbatim}
begin Mesh

  ...

  dimension 3

  ...

  Nx  16
  Dx  0.0625

  Ny  16
  Dy  0.0625
	
  Nz  1
  Dz  -1.0 

  ...

end
\end{verbatim}
%
Observe that the $z$-direction element size must be negative. This is
a consequence of the origin being at the top corner of the domain.

Also, if the mesh dimension is less than 3, only the element sizes in
the ``missing'' dimensions are specified and not the number of
elements. The sizes must be given to ensure consistent volumes and
areas. For a 1D domain, the sizes would be given as follows:
%
\begin{verbatim}
begin Mesh

  ...

  dimension 1

  ...

  Nx  100
  Dx  10.0

  Dy  2.0

  Dz  -0.1

  ...

end
\end{verbatim}


\subsubsection{Rock region}
\label{sec:rock-region}

Region specific rock parameters can be given. These may include
relative permeability and capillary pressure properties on a regional
basis. For the uniform mesh, only a single region can be defined
covering the whole domain. The keyword \texttt{RockRegion} is followed
by the rock region name:
%
\begin{verbatim}
begin Mesh

  ...

  RockRegion  Stone

  ...

end
\end{verbatim}
%
The rock region name, \texttt{Stone} in the example above, must be
consistent rock region definitions given in the general part of the
simulator input file.

\subsubsection{Rock data}
\label{sec:rock-data}

Rock parameters such as porosity and permability is given in the
\texttt{RockData} subsection. Currently, the only legal type is
\texttt{Global}, which indicates that the parameters must be given for
the whole domain. Then, the required arrays are \texttt{poro},
\texttt{permx}, \texttt{permy}, and \texttt{permz}. The optional
off-diagonal permeability tensor elements are given by
\texttt{permxy}, \texttt{permxz} and \texttt{permyz}. If not
specified, these default to 0.0. The permeability values should be
given in $[\text{m}^2]$ ($1\: \text{m}^2 = 1.01325\cdot 10^{12}\:\text{Darcy}$).

\begin{verbatim}
begin Mesh

  ... 

  begin RockData

    type Global

    array poro
      0.20  0.40  0.13  0.25  0.30  0.35
    end

    array permx
      1.0
    end

    include y-and-z-perm.dat

  end

  ... 

end
\end{verbatim}
%
The arrays must each have either one value or as many values as there
are mesh elements. If only one value is given, this is used for all
the elements, see \texttt{permx} in the example above. 

The rock data section is also a typical place to use the
\texttt{include} keyword. This is illustrated above by including the
file ``y-and-z-perm.dat'' which contains:
%
\begin{verbatim}
  array permy
    100.0  105.0  200.0  250.0  0.1  4.0
  end

  array permz
    10.0  1.5  25.0  20.0  1.0  44.0
  end
\end{verbatim}


%%% Local Variables: 
%%% mode: latex
%%% TeX-master: t
%%% End: 
