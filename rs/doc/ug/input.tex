\chapter{Simulator input file specification}
\label{chapter:input}

\minitoc

The input file to the simulator defines all the parameters needed to
carry out a simulation, including fluid properties, rock data, the
geometrical domain, its discretisation into a mesh, and the fluid flux
discretisation.

%======================================================================

\csection{Configuration file overview}

The simulator accepts a single main input file, \texttt{run}, which is
organized into several sections. Broadly, the input file would look
like this:
\begin{verbatim}
begin RunSpec
  % Specifies numerical parameters and timestepping controls
end

begin Components
  % Distinct chemical species in the study
end

begin EquationOfState
  % Method for calculating phase properties and equilibrium
end

begin RockFluid
  % Relative permeability and capillary pressures
end

begin Sources
  % Fluid sources/sinks for driving the flow
end

begin InitialValues
  % Initial fluid distribution in the reservoir
end
\end{verbatim}

Since the input file may be rather large, it is recommended to split
it into several smaller files which are included from the main file.

%======================================================================

\csection{Rock/fluid properties}

Relative permeability and capillary pressures are given on a region
basis in the \texttt{RockFluid} section:
\begin{verbatim}
begin RockFluid

  begin SandStone
    type SomeRockFluid
    ...
  end

  begin Shale
    type AnotherRockFluid
    ...
  end

end
\end{verbatim}
The names of the subsections (here: \texttt{SandStone} and
\texttt{Shale}) must match the region names defined in the
\texttt{Mesh} section. A model must be specified for each rock/fluid
region, and two types are currently available.

%----------------------------------------------------------------------

\subsection{Twophase rock/fluid properties}

For simple studies with either two phases or with pairwise two phases,
the \text{type TwoPhaseRockFluid} may be suitable. It uses the
relative permeability functions
\begin{equation}
  k_r^\ell(S^\ell) = \left(\frac{S^\ell - S_r^\ell}{1-S_r^\ell}\right)^2.
\end{equation}
Residual saturations default to zero. Capillary pressures are given as
the functions $p_c^{ow}(S^w)$ and $p_c^{go}(S^g)$, and these also
default to zero. An example:
\begin{verbatim}
begin RockFluid

  begin SandStone
    type TwoPhaseRockFluid

    Swr  0.2 % Water residual saturation
    Sor  0.1 % Oil residual saturation
    Sgr  0.3 % Gas residual saturation

    begin pcow
      % 1D function of Sw
      ...
    end

    begin pcgo
      % 1D function of Sg
      ...
    end
  end

  ...

end
\end{verbatim}

%----------------------------------------------------------------------

\subsection{Tabular rock/fluid properties}

This model uses tables for all relative permeabilities, and an
interpolated relation for the oil relative permeability. For water and
gas the functions needed are $k_r^w(S^w)$ and $k_r^g(S^g)$, while for
oil, the functions $k_r^{ow}(S^o)$ and $k_r^{og}(S^o+S^w)$ are used.
Then
\begin{equation}
  k_r^o\left(S^w,S^o,S^g\right) =
  \frac{S^g k_r^{og}(S^o+S^w) + S^w k_r^{ow}(S^o)}{S^g + S^w}.
\end{equation}
Capillary pressures are given as $p_c^{ow}(S^w)$ and $p_c^{go}(S^g)$,
which default to zero. An example follows:
\begin{verbatim}
begin RockFluid

  begin Shale
    type TabularRockFluid

    begin krw
      % 1D function of Sw
      ...
    end

    begin krg
      % 1D function of Sg
      ...
    end

    begin krow
      % 1D function of So
    end

    begin krog
      % 1D function of So+Sw
    end

    begin pcow
      % 1D function of Sw
    end

    begin pcgo
      % 1D function og Sg
    end

  end

  ...

end
\end{verbatim}

%----------------------------------------------------------------------

\subsection{Three-phase tabular rock/fluid properties}

This model is similar to the \texttt{TabularRockFluid}, but its tables
are two-dimensional in water and gas saturations:
\begin{verbatim}
begin RockFluid

  begin Clay
    type ThreePhaseTabularRockFluid

    begin krw
      % 2D function of (Sw,Sg)
      ...
    end

    begin krg
      % 2D function of (Sw,Sg)
      ...
    end

    begin kro
      % 2D function of (Sw,Sg)
      ...
    end

    begin pcow
      % 2D function of (Sw,Sg)
    end

    begin pcgo
      % 2D function of (Sw,Sg)
    end

  end

  ...

end
\end{verbatim}
Capillary pressures need not be given, as they default to zero.

%======================================================================

\csection{Components}

The section \texttt{Components} lists all the available components.
Each component is listed in a subsection, where the section name is
the component name. The following properties may be given in each
component subsection:

\begin{tabular}{lll}
  \texttt{Mw} & Molecular weight & $\left[\kilogram\per\mole\right]$ \\
  \texttt{Tc} & Critical temperature & $\left[\kelvin\right]$ \\
  \texttt{Pc} & Critical pressure & $\left[\pascal\right]$ \\
  \texttt{Vc} & Critical molar volume & $\left[\meter\cubed\per\mole\right]$ \\
  \texttt{omega} & Acentric factor & $\left[-\right]$
\end{tabular}

Default values are provided for H$_2$O, N$_2$, CO$_2$, H$_2$S, and the
twenty first alkanic hydrocarbons (C$_1$ through C$_{20}$). Example:

\begin{verbatim}
begin Components
  begin H2O
    Mw 0.1 % Modified weight
  end

  begin C1
    % Use defaults
  end

  begin C2
    % Use defaults
  end
end
\end{verbatim}

Notice that the water component (H$_2$O) will always be included, even
if omitted from the input file.

%----------------------------------------------------------------------

\csection{Equation of state}

At present, the simulator uses a either a black-oil equation of state
or a general cubic equation of state.

%......................................................................

\subsection{Black-oil}

In the black-oil model, it is necessary to group components into an
oil group and a gas group. Binary mixture thermodynamics are then used
to derive phase properties. This is accomplished using tables on molar
densities, molar enthalpies, viscosities, and dew/bubble-points:

\begin{tabular}{lll}
  \texttt{WaterMolarDensity} & $\xi^w(p,T)$ &
  $[\mole\per\meter\cubed]$ \\
  \texttt{OilMolarDensity} & $\xi^o\left(p,C_g^o,T\right)$ &
  $[\mole\per\meter\cubed]$ \\
  \texttt{GasMolarDensity} & $\xi^g\left(p,C_o^g,T\right)$ &
  $[\mole\per\meter\cubed]$ \\
  \texttt{WaterMolarEnthalpy} & $\bar h^w(p,T)$ &
  $[\joule\per\mole]$ \\
  \texttt{OilMolarEnthalpy} & $\bar h^o\left(p,C_g^o,T\right)$ &
  $[\joule\per\mole]$ \\
  \texttt{GasMolarEnthalpy} & $\bar h^g\left(p,C_o^g,T\right)$ &
  $[\joule\per\mole]$ \\
  \texttt{WaterViscosity} & $\mu^w(p,T)$ &
  $[\pascal\usk\second]$ \\
  \texttt{OilViscosity} & $\mu^o\left(p,C_g^o,T\right)$ &
  $[\pascal\usk\second]$ \\
  \texttt{GasViscosity} & $\mu^g\left(p,C_o^g,T\right)$ &
  $[\pascal\usk\second]$ \\
  \texttt{BubblePoint} & $C_{g,B}^o(p,T)$ & $\left[-\right]$ \\
  \texttt{DewPoint} & $C_{g,D}^o(p,T)$ & $\left[-\right]$
\end{tabular}

For iso-thermal runs the enthalpy tables may be omitted, and the
miscibility tables (bubble- and dew-points) default to zero (no
mixing). Example:

\begin{verbatim}
begin EquationOfState
  type BlackOilEquationOfState

  % Oil group
  array oil
    C3 C4
  end

  % Gas group
  array gas
    C1 C2
  end

  begin WaterMolarDensity
    % 2D function
    ...
  end

  begin WaterViscosity
    % 2D function
    ...
  end

  begin OilMolarDensity
    % 3D function
    ...
  end

  begin GasMolarDensity
    % 3D function
    ...
  end

  begin OilViscosity
    % 3D function
    ...
  end

  begin GasViscosity
    % 3D function
    ...
  end

end
\end{verbatim}

%......................................................................

\subsection{Cubic equation of state}

The general cubic equation of state is
\begin{equation}
  p = \frac{RT}{\bar V^\ell - b^\ell} - \frac{a^\ell}
  {\left(\bar V^\ell - \delta_1 b^\ell\right)
    \left(\bar V^\ell - \delta_2 b^\ell\right)}.
\end{equation}
$\bar V^\ell$ is the molar phase volume, and $a^\ell$ and $b^\ell$ are
mixing parameters, given in the Technical Description. The parameters
$\delta_1$ and $\delta_2$ depend on the type of cubic equation of
state, and currently there are two choices: Soave-Redlich-Kwong and
Peng-Robinson.

These are used for calculating the oil/gas properties and equilibrium,
while the water properties are calculated as for the black-oil case
with tables:

\begin{verbatim}
begin EquationOfState
  type CubicEquationOfState

  begin WaterMolarDensity
    % 2D function
    ...
  end

  begin WaterViscosity
    % 2D function
    ...
  end

  % PR for Peng-Robinson (default), SRK for Soave-Redlich-Kwong
  EOS PR

  % When the largest component fugacity difference is less than this (in Pascal),
  % switch from the sucessive substitution method to Newton's method
  FlashSwitchingCriteria 1e+4 % default value

  % EOS specific component data
  begin ComponentData
    begin C1
      s 0.1 % Volume shift, defaults to zero

      % Binary interaction coefficients, defaults to zero
      begin Binary
        C2 0.02 % Interaction between C1 and C2
        C3 0.01 % Interaction between C1 and C3
        ...
      end
    end

    begin C2
      ...
    end

    ...
  end

  % Coefficients for the Lohrenz-Bray-Clark viscosity calculations
  % Values are defaulted unless specified
  begin Viscosity
    a1 0.1023
    a2 0.023364
    a3 0.058533
    a4 -0.040758
    a5 0.0093324
  end
end
\end{verbatim}
Consult the Technical Description for details on the algorithm used
and the precise meaning of all the parameters.

%======================================================================

\csection{Initial values}

Initial temperature, $T$, and overall mass fractions, $C_\nu$, may be
specified as a function of the $z$-coordinate. A datum depth
($z$-coordinate) and pressure must also be given:

\begin{verbatim}
begin InitialValues

  DatumDepth    50   % z-coordinate
  DatumPressure 2e+7 % Pascal

  begin Temperature
    % 1D function of the z-coordinate, in Kelvin
    ...
  end

  begin H2O
    % 1D function of the z-coordinate, in moles
    ...
  end

  begin C1
    % 1D function of the z-coordinate, in moles
    ...
  end

end
\end{verbatim}

The temperature may not be omitted, but any of the mole fractions may,
in which case they default to zero. If the fractions do not sum to
unity, they are automatically normalized. If no fractions have been
given, then water is assumed: $C_{\text{H}_2\text{O}} = 1$.

%======================================================================

\csection{Sources}

The system is usually driven by the fluid sources and sinks. These
must already have been mapped by the mesh generator, as described in
Section~\ref{sec:gg:sources}. The \texttt{run} file adds information
on their behavior. There are three types: fixed, regular, and outlet.

\begin{verbatim}
begin Sources

  begin Fixed
    begin MyFixedSource1
      ...
    end

    begin MyFixedSource2
      ...
    end
  end

  begin Regular
    begin MyRegularSource
      ...
    end
  end

  begin Outlet
    begin MyProductionWell
      ...
    end
  end
end
\end{verbatim}

Note that the names must exactly match the names given in the
\texttt{mesh} file.

%----------------------------------------------------------------------

\subsection{Fixed source}

A fixed source has the same system state for all times. It is
specified by a temperature, $T$, and an overall composition, $C_\nu$.
Pressure, $p$, and total masses, $N$, are determined by a volume
balance method. Fractions are automatically normalized, and water is
substituted in the absence of any fractions.

\begin{verbatim}
begin Sources
  begin Fixed
    begin MyFixedSource
      Temperature 300

      % Component name / fraction
      H2O                0.1
      C1                 0.9

      ...
    end
    ...
  end
  ...
end
\end{verbatim}

%----------------------------------------------------------------------

\subsection{Regular source}

A regular source contributed energy, $q^e$, and molar masses,
$q^\nu$ to the associated elements. It is given by:
\begin{verbatim}
begin Sources
  begin Regular
    begin MyRegularSource
      % Watts
      Energy  1e+10

      % Component name / mol/s
      H2O                1e+10
      ...
    end
    ...
  end
  ...
end
\end{verbatim}
The energy and mass sources may be timedependent. Then a
one-dimensional function can be substituted with the appropriate name.
Its argument is the current time relative to the total time of the
run, hence between zero and one.

%----------------------------------------------------------------------

\subsection{Outlet}

An outlet has a specified energy sink, $q^e$, and calculates molar
mass losses by $q^\nu=\sum_\ell C_\nu^\ell f^\ell q_T$, where $f^\ell$
is the fractional flow function and $q_T$ is the given total molar
mass sink.
\begin{verbatim}
begin Sources
  begin Outlet
    begin MyOutlet
      % Watts
      Energy  -1e+10

      % mol/s
      Mass    -1e+10

      ...
    end
    ...
  end
  ...
end
\end{verbatim}
An outlet reports on the amount of mass and energy flowing through.  A
file with the outlet name will be created in the \texttt{simulation}
directory, containing production statistics.

As for the regular source, the energy and mass can be given as
functions of the dimensionless time (current time relative to the
total runtime).

%======================================================================

\csection{Run specification}

The run specification deals with timestepping, reporting and numerical
discretisation. It contains a large number of parameters, most of
which are optional. The following may be given within the section
\texttt{RunSpec}:

\begin{list}{}{}
\item[\texttt{TimeUnit}] Possible timeunits are \texttt{Seconds}
  (default), \texttt{Days}, \texttt{Years}, and \texttt{MillionYears}.
\item[\texttt{EndTime}] Endtime of the run in the chosen timeunit.
  Defaults to zero, so to enable a run forward, it must be given as a
  positive number.
\item[\texttt{MinimumTimeStep}] Smallest timestep size in the chosen
  timeunit. Defaults to $10^{-10}$.
\item[\texttt{MaximumTimeStep}] Largest timestep size in the chosen
  timeunit. Defaults to $1$.
\item[\texttt{ReportTimes}] Array of desired report times, in the
  chosen timeunit.
\item[\texttt{ReportEvery}] Evenly spaced report time, in the chosen
  time units. This is in addition to \texttt{ReportTimes}.
\item[\texttt{ReportAlways}] If true, every single timestep will be
  reported. Useful for debugging, hence it defaults to false.
\item[\texttt{Thermal}] True to enable thermal features, defaults to
  false.
\item[\texttt{MaximumNonlinearPressureIterations}] The maximum number
  of nonlinear pressure iterations to use. Defaults to $10$.
\item[\texttt{PressureTolerance}] Convergence criteria in the
  Newton-Raphson iteration on the volume balance equation. The
  iteration is finished once
  \begin{equation}
    \max_i \left|\frac{\Delta p_i}{p_i}\right| < \text{PressureTolerance}.
  \end{equation}
  Default value is $10^{-6}$. $\Delta p_i$ is the pressure change from
  the linear pressure solver, and $p_i$ is the current pressure, both
  in element $i$.
\end{list}

%----------------------------------------------------------------------

\csubsection{Timestep control}

There is also a set of parameters controlling the selection of the
timestep size, $\Delta t$. Firstly,
\begin{equation}
  \texttt{MinimumTimeStep} < \Delta t < \texttt{MaximumTimeStep}.
\end{equation}
Then
\begin{equation}
  \Delta t^{n+1} = \Delta t^n \min_i \left(
    \frac{(1+\lambda) \Delta x_T}{\Delta x + \lambda \Delta x_T}.
  \right),
\end{equation}
where $\lambda$ is given by \texttt{Lambda}, default is $0.5$, $\Delta
x$ is a quantity change, and $\Delta x_T$ is the target change. The
following target changes are monitored:
\begin{list}{}{}
\item[\texttt{TargetResidualVolume}] Defaults to $0.01$, and is
  computed as
  \begin{equation}
    \Delta R = \max_i \left|\frac{R_i}{V_i^p}\right|.
  \end{equation}
\item[\texttt{TargetNonLinearIterations}] Defaults to $0.5$, and is
  the target ratio of the number of nonlinear pressure iterations
  relative to the maximum allowed.
\item[\texttt{TargetThroughputRatio}] Defaults to $0.5$, and is
  the target ratio of the total molar change in a grid block relative
  to the amount present in the beginning of a timestep:
  \begin{equation*}
    \Delta T = \max_i \frac{\Delta N_i}{N_i},\quad N_i>0.
  \end{equation*}
\end{list}

%----------------------------------------------------------------------

\csubsection{Linear solver}

The pressure and temperature system is stored in a sparse matrix, and
for parallel runs, this matrix system is automatically distributed
amongst the participating threads, see Figure~\ref{fig:matrix}. A
Krylov subspace solver is used, and a block diagonal preconditioner
may be optionally applied. The blocks are the $A_i$ matrices in the
figure.

\begin{figure}
  \centering
  \begin{picture}(100,100)
    \thicklines
    \put(0,0){\framebox(100,100){}}

    \thinlines
    \put(0,75){\dashbox{5}(25,25){$A_1$}}
    \put(25,50){\dashbox{5}(25,25){$A_2$}}
    \put(50,25){\dashbox{5}(25,25){$A_3$}}
    \put(75,0){\dashbox{5}(25,25){$A_4$}}

    \put(0,75){\dashbox{5}(100,0){}}
    \put(0,50){\dashbox{5}(100,0){}}
    \put(0,25){\dashbox{5}(100,0){}}

    \put(25,75){\makebox(75,25){$B_1$}}
    \put(50,50){\makebox(50,25){$B_2$}}
    \put(0,25){\makebox(50,25){$B_3$}}
    \put(0,0){\makebox(75,25){$B_4$}}
  \end{picture}
  \caption{Matrix structure with four subdomains. Thread $i$ stores
    the block diagonal part $A_i$ and the off-diagonal part $B_i$. The
    submatrices $A_i$ and $B_i$ are stored in a compressed row format
    \cite{templates}.}
  \label{fig:matrix}
\end{figure}

There are quite a number of options, all are optional:
\begin{list}{}{}
\item[\texttt{NumberOfIterations}] Maximum number of iterations
  (default: 1000).
\item[\texttt{RelativeTolerance}] Relative convergence tolerance
  (default: $10^{-50}$).
\item[\texttt{AbsoluteTolerance}] Absolute convergence tolerance
  (default: $10^{-15}$).
\item[\texttt{DivergenceTolerance}] Divergence tolerance (default:
  $10^{5}$).
\item[\texttt{ReportIterations}] If set to ``yes'', the residual at
  each step of the iterative process will be printed.
\item[\texttt{LinearSolver}] The name of the Krylov subspace
  method. Possibilities are:
  \begin{list}{}{}
  \item[\texttt{BiCG}] BiConjugate Gradients.
  \item[\texttt{BiCGstab}] BiConjugate Gradients stabilized (default).
  \item[\texttt{CG}] Conjugate Gradients.
  \item[\texttt{CGS}] Conjugate Gradients Squared.
  \item[\texttt{GMRES}] Restarted generalized minimal residual.
  \item[\texttt{IR}] Iterative refinement.
  \item[\texttt{QMR}] Quasi minimal residual.
  \end{list}
\item[\texttt{Preconditioner}] Associated
  preconditioner. Possible choices are:
  \begin{list}{}{}
  \item[\texttt{Diagonal}] Inverse of the matrix diagonal.
  \item[\texttt{SSOR}] Successive overrelaxation.
  \item[\texttt{ICC}] Incomplete Cholesky decomposition.
  \item[\texttt{ILU}] Incomplete LU factorization (default).
  \item[\texttt{ILUT}] Incomplete LU with thresholding.
  \item[\texttt{AMG}] Algebraic multigrid using smoothed aggregation.
  \item[\texttt{none}] No preconditioning.
  \end{list}
\end{list}

%%% Local Variables: 
%%% mode: latex
%%% TeX-master: "ug"
%%% End: 
