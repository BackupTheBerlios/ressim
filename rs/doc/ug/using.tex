\chapter{Using the simulation suite}

\minitoc

The simulator is written in the Java programming language \cite{java},
and hence should be usable on most platforms. It requires at least
version 1.5 of the Java development environment.

%======================================================================

\csection{Retrieving and building the code}

The simulator code is stored in a CVS (concurrent version system)
repository on the host \texttt{rs.cipr.uib.no} (an alias for the
web-hotel \texttt{hilton.uib.no}). You can get an account on that
machine by inquiring on \texttt{bs.uib.no}.

The CVS repository root is \texttt{/home/www/rs.cipr.uib.no/cvsroot},
and the project name is just \texttt{rs}. Once checked out, the
directory structure is as follows:

\begin{tabular}[c]{ll}
  \texttt{bin}    & Binaries (\texttt{.class} files) \\
  \texttt{dist}   & The packaged simulator code \\
  \texttt{data}   & Data for simulator testing \\
  \texttt{doc}    & \LaTeX\ documentation \\
  \texttt{ex}     & Examples \\
  \texttt{lib}    & Libraries used by the simulator \\
  \texttt{libsrc} & Source code of the libraries \\
  \texttt{src}    & Source code (\texttt{.java} files)
\end{tabular}

Once it has been checked out, it can be built by running the included
Ant-script\footnote{\texttt{http://ant.apache.org}}
\texttt{build.xml}. Several targets are supported, including:

\begin{tabular}[c]{ll}
  \texttt{build} & Compiles the simulator \\
  \texttt{clean} & Cleans out the binaries \\
  \texttt{jar} & Build \texttt{rs/dist/rs.jar} \\
  \texttt{javadoc} & Creates Javadoc in \texttt{rs/dist/doc}
\end{tabular}

For visualization, GMV\footnote{\texttt{http://laws.lanl.gov/XCM/gmv}}
has been used. It is freely available for numerous platforms.

The code has been developed using the
Eclipse\footnote{\texttt{http://www.eclipse.org}} development
environment.

%======================================================================

\csection{Running a simulation}

First, build the \texttt{rs.jar} file by issuing \texttt{ant jar}.
Then copy the resulting \texttt{rs/dist/rs.jar} file to somewhere
permanent. For convenience, make an alias like the following:
\begin{verbatim}
> alias rs="java -jar /absolute/path/to/rs.jar"
\end{verbatim}
Then typing just \texttt{rs} will start the simulator:
\begin{verbatim}
> rs

        CIPR in-house reservoir simulation suite

Usage:

        mesh     - Generate a mesh
        mesh_gmv - Visualize a mesh using GMV
        run      - Run a simulation
        pvt      - Calculate PVT properties
        upscale  - Perform upscaling
        run_gmv  - Visualize simulation results using GMV
\end{verbatim}

%----------------------------------------------------------------------

\csubsection{Generating a mesh}

Typing \texttt{rs mesh} will start the mesh generator. It reads the
input file \texttt{mesh}, whose contents are described in
Chapter~\ref{mesh-generators}:
\begin{verbatim}
> rs mesh

        CIPR in-house reservoir simulation suite

        Mesh generator

MeshGenerator: 3D Structured mesh generator
15006 points
72000 interfaces
33100 neighbour connections
0 non-neighbour connections
33100 connections
12000 elements
TransmissibilityMethod: Using multi-point flux approximation with continuity=1.0
\end{verbatim}
The resulting mesh may be inspected by first issuing \texttt{rs
  mesh\_gmv}:
\begin{verbatim}
> rs mesh_gmv

        CIPR in-house reservoir simulation suite

        GMV mesh export

15006 points
72000 interfaces
33100 neighbour connections
0 non-neighbour connections
33100 connections
12000 elements
\end{verbatim}
Then GMV may be started:
\begin{verbatim}
> cd visualization
> gmv -i mesh &
\end{verbatim}

%----------------------------------------------------------------------

\csubsection{Performing a simulation}

Once you're satisfied with the mesh, a simulation may be started by
typing \texttt{rs run}. The simulator then reads the \texttt{run}
file, described in Chapter~\ref{chapter:input}. For example:
\begin{verbatim}
> rs run

        CIPR in-house reservoir simulation suite

        Reservoir flow simulator

RunSpec: Simulating from 0.0 to 1.0 million years
RockFluid:Rock: Two phase properties
Components: (14) H2O C1 C10 C2 C3 C4 C5 C6 C7 C8 C9 CO2 H2S N2
EquationOfState: Using Peng-Robinson cubic equation of state
Sources:Fixed: outlet
Sources:Regular: source
Sources:Outlet:
Subdomains: 1
[1] 12000 elements

 ...

\end{verbatim}
Some information on the run is first listed, then the mesh size.
Messages starting with \texttt{[\#]}, where \# is a number, indicate
subdomain specific messages. Here, the number of connections and
elements on subdomain 1 is given. During the simulation, these
messages are printed at every timestep:
\begin{verbatim}
t = 1.300000e-01 (13%), dt = 8.294701e-03
  1 2.214939e-06
  2 5.029365e-08
R = 2.997585, NR = 1.666667, T = 2.999976. Limit: Pressure iterations
\end{verbatim}
Here, the current time is given, the percentage completed, and the
current timestep size. The next two lines given the pressure
iterations, with the iteration count and the maximum relative pressure
change that iteration. The last gives information on how the timestep
will be subsequently modified. \texttt{R} refers to residual volume,
\texttt{NR} to the number of Newton-Raphson pressure iterations, and
\texttt{T} to the throughput.  Numbers larger than one indicates that
the timestep should increase, while below indicates a decrease. The
most conservative is always chosen.

It is possible to restart a simulation from a saved state. The stored
states are located in the \texttt{simulation} directory:
\begin{verbatim}
> ls simulation
0.000000  0.130000  0.260000  0.390000  0.520000  0.650000  0.770000 ...
\end{verbatim}
You may then restart by issuing for instance
\begin{verbatim}
> rs run 0.770000

        CIPR in-house reservoir simulation suite

        Reservoir flow simulator

RunSpec: Simulating from 0.77 to 1.0 million years
...
\end{verbatim}
which continues from the given time.

After, or even during, a simulation, the results may be inspected. To
do this, simply issue \texttt{rs run\_gmv}:
\begin{verbatim}
> rs run_gmv

        CIPR in-house reservoir simulation suite

        GMV field export

15006 points
72000 interfaces
33100 neighbour connections
0 non-neighbour connections
33100 connections
12000 elements
Reading 10 fields .......... done
Removing 10 files .......... done
Writing output files .......... done
\end{verbatim}
Then start the GMV application:
\begin{verbatim}
> cd visualization
> gmv -i 0000 &
\end{verbatim}

%----------------------------------------------------------------------

\csubsection{Upscaling and PVT calculations}

Upscaling is done by issuing \texttt{rs upscale}. It reads the file
\texttt{upscale}, and requires that a mesh has been generated. Outputs
upscaled permeabilities and porosities.

\begin{verbatim}
> rs upscale

        CIPR in-house reservoir simulation suite

        Absolute permeability upscaling

symmetry: k12 = 1.7081674899790647E-10, k21 = 1.7153376620723159E-10
symmetry: using kxy = kyx = k12
eigenvalues: 8.732770628490563 50.50000000170697
\end{verbatim}

PVT properties can be investigated by \texttt{rs pvt}, which reads the
file \texttt{pvt}. The content of this file is described in
Chapter~\ref{chapter:pvt}. Example:

\begin{verbatim}
> rs pvt

        CIPR in-house reservoir simulation suite

        PVT property calculation

Components: (14) H2O C1 C10 C2 C3 C4 C5 C6 C7 C8 C9 CO2 H2S N2
EquationOfState: Using Peng-Robinson cubic equation of state

---------------------------------------
        PVT properties
---------------------------------------

Phase                     WATER
...

Phase                       OIL
...

Phase                       GAS
...
\end{verbatim}

%======================================================================

\csection{Examples}

The \texttt{ex} directory contains several examples. These include:

\begin{tabular}[c]{ll}
  \texttt{bl} & Buckley-Leverett \\
  \texttt{migbench} & Hydrocarbon migration \\
  \texttt{mpfa} & Skew grid test \\
  \texttt{qfive} & Quarter-of-a-fivespot, several variants \\
  \texttt{qfivefrac} & Quarter-of-a-fivespot on a fractured medium \\
  \texttt{tri} & Triangular grids
\end{tabular}

The \texttt{ex/pvt} directory contains case-independent PVT files.

%%% Local Variables: 
%%% mode: latex
%%% TeX-master: "ug"
%%% End: 
