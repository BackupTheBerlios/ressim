\chapter{Mesh specification}
\label{chapter:mesh}

\minitoc

The simulator accepts a single unstructured mesh format which is
defined within the section \texttt{Mesh}. This section is typically
not edited by the user, although minor modifications can be done by
hand. Instead, this section should be created with either the
accompanying mesh generator or constructed from other external meshing
tools. The \texttt{Mesh} section must contain subsections describing
geometry and data mappings. Subsections for choosing transmissibility
computation methods are optional:
%
\begin{verbatim}
begin Mesh

  begin Points
    ...
  end

  begin Interfaces
    ...
  end

  begin Elements
    ...
  end

  begin NeighbourConnections
    ...
  end

  % optional
  begin NonNeighbourConnections
    ...
  end

  begin RockRegionMap
    ...
  end

  begin RockData
    ...
  end

  begin AbsPermTransmissibilities
    ...
  end

  % for thermal runs only
  begin CondTransmissibilities
    ...
  end

end
\end{verbatim}
%
Note that in this native format all indices start at 0. 

\csection{Geometry}
\label{sec:geometry}

The mesh geometry is specified within the three subsections
\texttt{Points}, \texttt{Interfaces} and \texttt{Elements}.  Element
connections are listed in the \texttt{NeighbourConnections} and
\texttt{NonNeighbourConnections} subsections.

\subsection{Points}
\label{sec:points}
In the \texttt{Points} subsection the number of mesh points must be
specified along with the coordinates of each point. The order of the
coordinates implicitly determines the point numbering. Example:
%
\begin{verbatim}
  begin Points
    NumberOfPoints 2

    array Coordinates
      0.0 0.0 0.0     % point 0
      1.0 0.0 0.0     % point 1
    end
  end
\end{verbatim}

\subsection{Interfaces}
\label{sec:interfaces}
In the \texttt{Interfaces} subsection the number of mesh interfaces
must be specified. Each interface is specified in the \texttt{Points}
array. First the number of points for the interface is given, then the
indices to the contributing points are listed. These points must be
given in a circular order that results in an outward pointing normal
following the right hand rule. This is meaningful since interfaces are
not shared among elements of the mesh. The numbering of the interfaces
are given implictly by the order in the \texttt{Points} array. In
addition to the points, the interface areas, unit normals and center
point coordinates must be given for each interface.  Example:
%
\begin{verbatim}
  begin Interfaces
    NumberOfInterfaces 2

    array Points
      3   0 1 2     % interface 0 has 3 points: {0, 1, 2}
      4   2 1 3 4   % interface 1 has 4 points: {2, 1, 3, 4}
    end

    array CenterPoints
      0.0 0.0 0.0   % center point of interface 0
      0.5 0.0 0.5   % center point of interface 1
    end             
                    
    begin Areas     
      1.0           % area of interface 0
      2.0           % area of interface 1
    end             
                    
    begin Normals   
      0.0 0.0 1.0   % unit normal vector of interface 0
      0.0 1.0 0.0   % unit normal vector of interface 1
    end

  end
\end{verbatim}

\subsection{Elements}
\label{sec:elements-1}
In the \texttt{Elements} subsection the number of mesh elements (grid
cells) must be specified. Each element is given as a list of defining
interfaces. These are given in the \texttt{Interfaces} array which
lists the number of interfaces and interfaces indices for each
element. The ordering of elements are given implicitly from the
ordering of the \texttt{Interfaces} array. Additional data necessary
are the volumes and center point coordinates for each element.
Example:
%
\begin{verbatim}
  begin Elements

    NumberOfElements 3

    array Interfaces
      6   0 1 2 3 4  % element 0 has 6 interfaces: {0, 1, 2, 3, 4}
      4   1 2 5 6    % element 1 has 4 interfaces: {1, 2, 5, 6}
      5   3 4 6 7    % element 2 has 5 interfaces: {3, 4, 6, 7}
    end

    array Volumes
      1.0            % volume of element 0
      1.5            % volume of element 1
      20.0           % volume of element 2
    end

    array Centers
       0.0 0.0 0.0   % center point of element 0
       1.5 0.0 0.2   % center point of element 1
       0.0 1.0 0.5   % center point of element 2
    end

  end
\end{verbatim}


\subsection{NeighbourConnections}
\label{sec:neigh-connections}
In the \texttt{NeighbourConnections} subsection the number of
(internal) connections between neighbouring elements must be
specified. For each connection two interface indices must be given.
These are listed in the \texttt{Connections} array. Example:
%
\begin{verbatim}
  begin NeighbourConnections

    NumberOfConnections 2

    array Connections
      1 11    % neighbour connection 0 between interfaces 1 and 11
      7 9     % neighbour connection 1 between interfaces 7 and 9
    end

    % optional flux multipliers
    array FluxMultipliers
      0.9     % flux multiplier for neighbour connection 0
      0.7     % flux multiplier for neighbour connection 1
    end

  end
\end{verbatim}
%
An optional array of \texttt{FluxMultipliers} can be specified. If not
given, the flux multipliers all default to \texttt{1.0}.

\subsection{NonNeighbourConnections}
\label{sec:non-neigh-connections}
The \texttt{NonNeighbourConnections} subsection is optional. Here the
number of connections between non-neighbouring elements must be
specified. For each connection two element indices must be given.
These are listed in the \texttt{Connections} array. Example:
%
\begin{verbatim}
  begin NonNeighbourConnections

    NumberOfConnections 2

    array Connections
      0 4    % non-neighbour connection 0 between elements 0 and 4
      3 5    % non-neighbour connection 1 between elements 3 and 5
    end

    % optional flux multipliers
    array FluxMultipliers
      0.7     % flux multiplier for non-neighbour connection 0
      0.5     % flux multiplier for non-neighbour connection 1
    end

  end
\end{verbatim}
%
An optional array of \texttt{FluxMultipliers} can be specified. If not
given, the flux multipliers all default to \texttt{1.0}.


%%%%%%%%%%%%%%%%%%%%%%%%%%%%%%%%%%%%%%%%%%%%%%%%%%


\csection{Data mapping}
\label{sec:data-mapping}

Rock/fluid regions and rock data are mapped to elements in the
subsections \texttt{RockRegionMap} and \texttt{RockData}.

\subsection{RockRegionMap}
\label{sec:rockregionmap}
%
In the \texttt{RockRegionMap} subsection rock regions are mapped to
elements. For each region that should be mapped, an array of element
indices must be given. The array names must match names of regions
defined in the \texttt{RockFluid} section. All elements of the mesh
must be given a rock region. If an element occurs in more than one
region array, behaviour is undefined. An example of the format for a
mesh of 5 elements with the regions \texttt{SandStone} and
\texttt{Shale} defined:
%
\begin{verbatim}
  begin RockRegionMap
 
    array Shale
      0 2 4
    end

    array SandStone
      1 3 5
    end

  end
\end{verbatim}



\subsection{RockData}
\label{sec:rockdata}

In the \texttt{RockData} subsection, porosity, permeability,
conductivity, compressibility and heat capacity can be given for each
element. Porosity, \texttt{Poro}, and diagonal permeabilities,
\texttt{PermX}, \texttt{PermY}, \texttt{PermZ}, are required. Thermal
runs also require the rock heat capacity, \texttt{C}, and heat
conductivity tensor diagonals, \texttt{CondX}, \texttt{CondY},
\texttt{CondZ}. Optional data are the off-diagonal permeabilities,
\texttt{PermXY}, \texttt{PermXZ}, \texttt{PermYZ}, the off-diagonal
rock heat conductivities, \texttt{CondXY}, \texttt{CondXZ},
\texttt{CondYZ}, and the rock compressibility, \texttt{Cr}. The array
values follow the element ordering, and the number of values must
equal the number of elements. Otherwise, if an array contains only one
value, that value is used for all elements. Example for a mesh with 5
elements:
%
\begin{verbatim}
  begin RockData

    begin Poro
      0.2 0.25 0.3 0.15 0.2
    end
  
    begin PermX
      1.0e-5        % one value is used for all elements      
    end
  
    begin PermY
      1.0e-5  5.0e-6  6.0e-6  1.0e-5  1.0e-5  9.0e-8
    end
  
    begin PermZ
      include permz.dat  % file permz.dat contains 5 permeability values
    end

  end
\end{verbatim}
%

\csection{Transmissibilities}
\label{sec:transmissibilities}

Absolute permeability and heat conductivity transmissibilities are
computed externally. The latter is only relevant for thermal runs.

\subsection{AbsPermTransmissibilities}
\label{sec:absp}
Transmissibilities must be specified for all connections, neighbouring
and non-neighbouring (if present). For each connection, two
transmissibility-values and two element-indices are required.
Additional transmissibility values and corresponding element indices
can be given for sub-connections of multi-point flux stencils.
Example:
%
\begin{verbatim}
    begin AbsPermTransmissibilities

      % mesh connection 0       
      begin 0
        array Transmissibilities
          1.0 -1.0
        end
      
        array Elements
          0 1
        end
      end

      % mesh connection 1
      begin 1                    
        array Transmissibilities
          0.5 -0.5
        end
      
        array Elements
          1 2
        end
      end

      ...

    end
\end{verbatim}
%

\subsection{CondTransmissibilities}
\label{sec:condtr}

The format is the same as for the absolute permeability
transmissbilities above. This section is only used for thermal runs.
%
\begin{verbatim}
  begin CondTransmissibilities

    % same syntax as AbsPermTransmissibilities
    ...

  end
\end{verbatim}
%

%%%%%%%%%%%%%%%%%%%%%%%%%%%%%%%%%%%%%%%%%%%%%%%%%%%%%%%%%%%%


%%% Local Variables: 
%%% mode: latex
%%% TeX-master: "ug"
%%% End: 
