\chapter{Configuration file syntax}

The input files are written in a plain text format, where elements are
separated from each other by whitespace. Whitespace is one or more of:
space, newline, tab, or similar. Hence, whitespaces may not occur in
any identifier.  The file format is:
\begin{description}
\item[Key/value] The basic construct is the key=value pairing. Examples:
\begin{verbatim}
keyword1	value1
keyword2	value2
\end{verbatim}
\item[Arrays] Arrays of data may be entered as follows:
\begin{verbatim}
array myArray
  data1 data2 data3
end
\end{verbatim}
  Strings, integers, and floating point numbers may be used in arrays.

\item[Nested includes] Another configuration file may be included into
the current file:
\begin{verbatim}
include myFile
\end{verbatim}
Neither the filename nor its path may include whitespaces. By default
files are located relative to the current directory. If the keyword 
{\tt rinclude} is used instead of {\tt include}, files are located relative
to the file in which the directive is given. This may be useful for
files that need to include further files only knowing where they are
located relative to themselves.
\item[Sections] The configuration file is divided into a hierarchy of sections,
as illustrated here:
\begin{verbatim}
begin mySection
  ... section data ...
end
\end{verbatim}
The section data may include all the usual configuration file contents.
\end{description}

The order of the different elements do not matter, as the input files
are in a free-format form.

Comments can be written anywhere on a line after the comment sign
``\%''. What's written after ``\%'' until the end-of-line is skipped.

%----------------------------------------------------------------------

\csection{Lookup table syntax}

Lookup tables are common elements in a configuration file. There are two basic
types: even lookup tables which has regularly spaced data points, and standard
lookup tables with irregularly spaced points. Each of these tables may be
given in 1D, 2D, or 3D.

%......................................................................

\subsubsection{1D even lookup table}

\begin{verbatim}
begin MyEven1DLookupTable
  type EvenLookupTable

  dimension 1

  % Data given at x={0,1/3,2/3,1}
  xMin 0
  xMax 1
  nx   4

  array data
    1 % x=0
    2 % x=1/3
    3 % x=2/3
    4 % x=1
  end
end
\end{verbatim}

%......................................................................

\subsubsection{1D lookup table}

\begin{verbatim}
begin My1DLookupTable
  type LookupTable

  dimension 1

  % X coordinates
  array coord1
    0 0.4 0.8 1
  end

  array data
    1 % x=0
    2 % x=0.4
    3 % x=0.8
    4 % x=1
  end
end
\end{verbatim}

%......................................................................

\subsubsection{2D even lookup table}

\begin{verbatim}
begin MyEven2DLookupTable
  type EvenLookupTable

  dimension 2

  % Data given at x={0,1/2,1} ...
  xMin 0
  xMax 1
  nx   3

  % ... and at y={2,3,4}
  yMin 2
  yMax 4
  ny   3

  array data
    % y=2
    % x=0, 1/2, 1
        1	  2   3

    % y=3
    % x=0, 1/2, 1
        4	  5   6

    % y=4
    % x=0, 1/2, 1
        7	  8   9
  end
end
\end{verbatim}

%......................................................................

\subsubsection{2D lookup table}

\begin{verbatim}
begin My2DLookupTable
  type LookupTable

  dimension 2

  % X coordinates
  array coord1
    0 0.5 0.7 1
  end

  % Y coordinates
  array coord2
    2 3 3.5 4
  end

  array data
    % y=2
    % x=0, 0.5, 0.7, 1
        1	  2    3   4

    % y=3
    % x=0, 0.5, 0.7, 1
        5	  6    7   8

    % y=4
    % x=0, 0.5, 0.7, 1
        9	  10   11  12
  end
end
\end{verbatim}

%......................................................................

\subsubsection{3D even lookup table}

\begin{verbatim}
begin MyEven3DLookupTable
  type EvenLookupTable

  dimension 3

  % Data given at x={0,1/2,1} ...
  xMin 0
  xMax 1
  nx   3

  % ... and at y={2,3,4} ...
  yMin 2
  yMax 4
  ny   3

  % ... and at z={10,20}
  zMin 10
  zMax 20
  nz   2

  array data
    % y=2, z=10
    % x=0, 1/2, 1
        1	  2   3

    % y=3, z=10
    % x=0, 1/2, 1
        4	  5   6

    % y=4, z=10
    % x=0, 1/2, 1
        7	  8   9

    % y=2, z=20
    % x=0, 1/2, 1
      1.5	 2.5  3.5

    % y=3, z=20
    % x=0, 1/2, 1
      4.5	 5.5  6.5

    % y=4, z=20
    % x=0, 1/2, 1
      7.5	 8.5  9.5
  end
end
\end{verbatim}

%......................................................................

\subsubsection{3D lookup table}

\begin{verbatim}
begin My3DLookupTable
  type LookupTable

  dimension 3

  % X coordinates
  array coord1
    0 0.5 0.7 1
  end

  % Y coordinates
  array coord2
    2 3 3.5 4
  end

  % Z coordinates
  array coord2
    10 20
  end
	
  array data
    % y=2, z=10
    % x=0, 0.5, 0.7, 1
        1	  2    3   4

    % y=3, z=10
    % x=0, 0.5, 0.7, 1
        5	  6    7   8

    % y=4, z=10
    % x=0, 0.5, 0.7, 1
        9	  10   11  12

    % y=2, z=20
    % x=0, 0.5, 0.7, 1
      1.5	 2.5  3.5  4.5

    % y=3, z=20
    % x=0, 0.5, 0.7, 1
      5.5	 6.5  7.5  8.5

    % y=4, z=20
    % x=0, 0.5, 0.7, 1
      9.5	 10.5 11.5 12.5
  end
end
\end{verbatim}

%======================================================================

\csection{Constants}

As an alternative to providing a lookup table, a single constant value
may be specified. Such a constant may be given quite briefly by
\begin{verbatim}
MyConstant 10
\end{verbatim}
where the function \texttt{MyConstant} is assigned the value 10. This
is equivalent to the more cumbersome
\begin{verbatim}
begin MyConstant
  type  ConstantValue
  value 10
end
\end{verbatim}

Constants may be supplied in place of any function, 1D, 2D, or 3D.

%%% Local Variables: 
%%% mode: latex
%%% TeX-master: "ug"
%%% End: 
