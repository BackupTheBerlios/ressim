\chapter{Fluid}
\label{chap:fluid}

\minitoc

The fluid package calculates and stores PVT properties for a general
compositional fluid. For higher performance, it also includes a
special black-oil model as well.

%======================================================================

\csection{Static fluid data}

The different fluid phases are enumerated:
\begin{lstlisting}
  enum Phase {
    WATER, OIL, GAS;
  }
\end{lstlisting}
Fluid components are instances of the \texttt{Component} class:
\begin{lstlisting}
  class Component {

    // Component index $\nu$
    int index();

    // Component name
    String name();

    // Component molecular weight $M_\nu$
    double getMolecularWeight();

    // Critical temperature $T_\nu^c$
    double getCriticalTemperature();

    // Critical pressure $p_\nu^c$
    double getCriticalPressure();;

    // Critical molar volume $\bar V_\nu^c$
    double getCriticalMolarVolume();

    // Pitzer's acentric factor $\omega_\nu$
    double getAcentricFactor();
  }
\end{lstlisting}
This class stores a small database of default values for many
components. The components are in turn collected into a component
database:
\begin{lstlisting}
  class Components {

    // All the stored components
    Component[] all;

    // The water component
    Component water();

    // All the components, except water
    Component[] hc;

    // Gets a component by name
    Component getComponent(String name);

    // Gets a component by index
    Component getComponent(int index);

    // Number of components
    int numComponents();
  }
\end{lstlisting}
The indices are set by the \texttt{Components} class. The water
component always gets index zero. Note that the use of primitive
arrays rather than collections is due to performance profiling.

Generic data may be associated with each phase:
\begin{lstlisting}
  class PhaseData<T> {
    T water; // Water property
    T oil;   // Oil property
    T gas;   // Gas property

    T get(Phase phase);
    void set(Phase phase, T value);
  }
\end{lstlisting}
A specialization, \texttt{PhaseDataDouble}, has been created for
higher performance when storing double-precision numbers.

A composition is used for storing global mass compositions $N_\nu$ or
phase compositions $N_\nu^\ell$:
\begin{lstlisting}
  class Composition {

    // Getters and setters
    double getMoles(Component nu);
    void setMoles(Component nu, double N);

    // Gets $C_\nu$ or $C_\nu^\ell$
    double getMoleFraction(Component nu);

    // Gets the total number of moles
    double getMoles();

    // Zeros out the composition
    void zero();
  }
\end{lstlisting}

%======================================================================

\csection{Equation of state calculations}

An equation of state represents the interface to a general PVT
calculation (black-oil, compositional, etc):

\begin{lstlisting}
  abstract class EquationOfState {
    abstract void calculatePhaseState(
      double p, Composition N, double T,
      PhaseData<EquationOfStateData> data);
  }
\end{lstlisting}
The global pressure (taken as the oil pressure), composition and
temperature are used to calculate the phase compositions, stored in an
auxiliary data object for each phase:

\begin{lstlisting}
  class EquationOfStateData {
    Phase phase;     // What phase is this?
    boolean present; // And is it present?

    double      V; // $V^\ell$
    Composition N; // $N_\nu^\ell$
    double     xi; // $\xi^\ell$

    double     dVdp; // $\left.\partial V^\ell/\partial p\right|_{T,N_\nu^\ell,\forall\nu}$
    double     dVdT; // $\left.\partial V^\ell/\partial T\right|_{p,N_\nu^\ell,\forall\nu}$
    double[]   dVdN; // $\left.\partial V^\ell/\partial N_\nu^\ell\right|_{p,T,N_\mu^\ell,\forall\mu\neq\nu}$

    double h;    // $\rho^\ell h^\ell$
    double dhdT; // $\left.\partial\left(\rho^\ell h^\ell\right)/\partial T\right|_{p,N_\nu,\forall\nu}$

    double mu; // $\mu^\ell$
  }
\end{lstlisting}

There are two implemented equations of state:
\begin{list}{}{}
\item[\texttt{BlackOilEquationOfState}] Uses tables for all
  properties, and in particular dew- and bubble-point tables. It
  performs very little data-verification, so care must be exersised
  when using it.
\item[\texttt{CubicEquationOfState}] Calculates oil/gas equilibrium
  based on equality of fugacities. Uses a mixed
  sucessive-subsitution/Newton's method approach.
\end{list}
Both inherit from \texttt{WaterEquationOfState}, which calculates the
properties of the water phase from tables.

%%% Local Variables: 
%%% mode: latex
%%% TeX-master: "im"
%%% End: 
