\chapter{Rock}
\label{chap:rock}

\minitoc

Rock data can be classified into two types: static rock data and
rock/fluid interaction data. The former is in general given on each
gridblock, and includes porosity, permeability, compressibility, etc.
The latter is given regionwise, and constitutes relative permeability
and capillary pressures.

All the rock data structures reside in the namespace
\texttt{no.uib.cipr.rs.rock}.

%======================================================================

\csection{Rock data}

The static rock data is stored in the Rock class:
\begin{lstlisting}
  class Rock {

    // Initial porosity $\phi_0$
    double getInitialPorosity();

    // Rock compaction, $c^r$
    double getRockCompaction();

    // Conductivity tensors
    Tensor3D getAbsolutePermeability(); // Absolute permeability, $\textbf K$
    Tensor3D getRockHeatConductivity(); // Rock heat conductivity, $\kappa$

    // Rock heat capacity, $\partial\left(\rho^r h^r\right)/\partial T$
    double getRockHeatCapacity();

    // Region name
    String getRegion();
  }
\end{lstlisting}
The region name may be used to determine the associated rock/fluid
properties for the host gridcell.

%======================================================================

\csection{Rock/fluid data}

Rock/fluid data is stored in the RockFluid class:
\begin{lstlisting}
  abstract class RockFluid {

    // Calculates $k_r^\ell$
    abstract void calculateRelativePermeability
      (PhaseDataDouble S, Element el, PhaseDataDouble kr);

    // Calculates $p_c^{ow}$
    abstract double calculateOilWaterCapillaryPressure
      (double Sw, Element el);

    // Calculates $p_c^{go}$
    abstract double calculateGasOilCapillaryPressure
      (double Sg, Element el);
  }
\end{lstlisting}
A reference to the actual element is included as some specific models
need to know the surrounding geometry.

%%% Local Variables: 
%%% mode: latex
%%% TeX-master: "im"
%%% End: 
