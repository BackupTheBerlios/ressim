\chapter{Mesh generators}
\label{chap:meshgen}

\minitoc

The mesh generator entry point is \texttt{meshgen.Main}. When run, it
does the following:
\begin{enumerate}
\item Generates a fine mesh.
\item Optionally creates a dual continuum.
\item Splits the fine mesh into submeshes for parallelization and
  upscaling.
\item Calculates transmissibilities for each submesh.
\end{enumerate}
The end result is stored in the \texttt{gridding} directory, where the
fine mesh is stored as \texttt{gridding/mesh}, and the submeshes in
the files \texttt{gridding/mesh.\#}.

%%% Local Variables: 
%%% mode: latex
%%% TeX-master: "im"
%%% End: 
