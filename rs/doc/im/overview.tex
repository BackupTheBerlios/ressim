\chapter{Overview}

\minitoc

The simulation suite is written in the Java programming language,
version 1.5. Its namespace is \texttt{no.uib.cipr.rs}, and since all
classes are stored in that namespace, it will omitted when the context
is clear.

The entry point of the simulator is the class
\texttt{no.uib.cipr.rs.Main}. Its only task is to delegate the
execution to the user-chosen task, and it is this class which is
called when the jar-file \texttt{rs.jar} is run.

%======================================================================

\csection{Primary subsystems}

At present, there are six subsystems available:
\begin{list}{}{}
\item[\texttt{geometry}] Mesh data structures and flux calculation. The mesh is
  assumed to be unstructured, with polyhedral cells connected by a
  general graph topology. Transmissibilities between the cells may be
  calculated by either a two- or a multi-point method. See
  Chapter~\ref{chap:geometry}.
\item[\texttt{meshgen}] Mesh generators and import filters. Can generate
  structured meshes and also locally refined ones. Import filters are
  available for some unstructured formats. See
  Chapter~\ref{chap:meshgen}.
\item[\texttt{rock}] Rock and rock/fluid data. Stores rock parameters for each
  cell, and calculates relative permeabilities and capillary
  pressures. See Chapter~\ref{chap:rock}.
\item[\texttt{fluid}] PVT (pressure, volume, temperature) calculations.
  Calculates phase states from pressure, temperature, and overall
  molar masses. Both black-oil and cubic equation of states are
  implemented. See Chapter~\ref{chap:fluid}.
\item[\texttt{field}] Simulation system state. Stores the state in each control
  volume and on each control surface of the mesh, and calculates
  secondary variables from the primary. Also performs initial
  equilibration of the reservoir. See Chapter~\ref{chap:field}.
\item[\texttt{numerics}] Flow simulator. The simulator uses the other packages,
  and thus models a compositional fluid on an unstructured grid with a
  control volume discretisation. See Chapter~\ref{chap:numerics}.
\item[\texttt{upscale}] Petrophysics upscaling. Calculates upscaled
  permeability and porosity from single-phase upscaling. See
  Chapter~\ref{chap:upscale}.
\item[\texttt{output}] Data export into the General Mesh Viewer (GMV) format.
  Either just the mesh or the simulation data can be dumped. See
  Chapter~\ref{chap:output}.
\end{list}
In addition, some common utilities are located within the
\texttt{util} namespace.

%----------------------------------------------------------------------

\csubsection{Paths and data exchange}

As mentioned in the Users Guide, the different parts of the simulation
suite creates directories for data exchange and storage. These are
always relative to the directory the simulator is started from, and
are defined in the \texttt{Paths} class:
\begin{list}{}{}
\item[\texttt{mesh}] Mesh generator input file.
\item[\texttt{run}] Simulator input file.
\item[\texttt{pvt}] PVT calculations input file.
\item[\texttt{upscale}] Upscaling input file.
\item[\texttt{gridding/}] Directory containing the output of the mesh
  generator:
  \begin{list}{}{}
  \item[\texttt{mesh}] Fine scale mesh. Binary serialization data.
  \item[\texttt{mesh.\#}] Subdomain meshes, which also contains source
    locations and subdomain couplings. Binary serialization data.
  \end{list}
\item[\texttt{simulation/}] Directory containing the output from the simulator:
  \begin{list}{}{}
  \item[\texttt{\textit{time/\#}}] Field state at the given time for
    the given subdomain. Binary serialization data.
  \item[\texttt{\textit{producer}.csv}] Production data in semi-colon
    separated format. May be imported into a spreadsheet.
  \item[\texttt{time}] Time and timestep information, in text format.
  \end{list}
\item[\texttt{visualization/}] Directory containing the output from
  the output filters. Contents vary, and the files are usually
  binaries.
\end{list}
The input files are all in plain text, while the files in the
subdirectories are usually only meant to be read by the simulator.

%======================================================================

\csection{Common utilities}

In \texttt{no.uib.cipr.rs.util}, there are some utilities used by many
parts of the code:
\begin{list}{}{}
\item[\texttt{Configuration}] Reads in a configuration text file, and
  stores it in a hashmap. Where the keys are the string names and the
  values are either numbers, arrays, or subsections. The subsections
  are actually configurations themselves, stored recursively.
\item[\texttt{Function}] General function, $f(\vec x)$. Some instances:
  \begin{list}{}{}
  \item[\texttt{LookupTable}] Lookup table for dimensions 1, 2, or 3.
    Arbitrary spacing is permitted, and linear interpolation is used.
  \item[\texttt{EvenLookupTable}] Lookup table where the spacing is
    even. This allows for a somewhat higher performance.
  \item[\texttt{ConstantValue}] Constant in all dimensions.
  \end{list}
\item[\texttt{Constants}] Physical constants.
\item[\texttt{Tolerances}] Some ``generic'' tolerance criteria.
\item[\texttt{Logger}] Output mechanism for a multi-threaded run.
\end{list}

%%% Local Variables: 
%%% mode: latex
%%% TeX-master: "im"
%%% End: 
