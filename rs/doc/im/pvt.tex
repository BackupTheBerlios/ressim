\chapter{Fluid}

\minitoc

The fluid module calculates and stores PVT properties for a general
compositional fluid.

%======================================================================

\csection{Static fluid data}

The different fluid phases are enumerated:
\begin{lstlisting}
  enum Phase {
    WATER, OIL, GAS;
  }
\end{lstlisting}
Fluid components are instances of the Component class;
\begin{lstlisting}
  class Component {
    String name; // Component name
    int index;   // Component index $\nu$

    double M; // Component molecular weight $M_\nu$

    double Tc; // Critical temperature $T_\nu^c$
    double pc; // Critical pressure $p_\nu^c$

    double omega; // Pitzer's acentric factor $\omega_\nu$

    double[] d; // Binary interaction coefficients $d_{ij}$
  }
\end{lstlisting}
which are collected in a component database:
\begin{lstlisting}
  class Components {
    Map<String, Component> mapComponent; // Name-based retrival
    Component[] indexComponent;          // Index-based retrival
  }
\end{lstlisting}

Generic data may be associated with each phase:

\begin{lstlisting}
  class PhaseData<T> {
    T water; // Water property
    T oil;   // Oil property
    T gas;   // Gas property

    T get(Phase phase);
    void set(Phase phase, T value);
  }
\end{lstlisting}

A composition is used for storing global mass compositions $N_\nu$ or
phase compositions $N_\nu^\ell$:

\begin{lstlisting}
  class Composition {
    double[] N;
  }
\end{lstlisting}

%======================================================================

\csection{Equation of state calculations}

An equation of state represents the interface to a general PVT
calculation (black-oil, compositional, etc):

\begin{lstlisting}
  interface EquationOfState {
    PhaseData<EquationOfStateData> calculatePhaseState(
      double p, Composition N, double T,
      PhaseData<EquationOfStateData> data);
  }
\end{lstlisting}
The global pressure (taken as the oil pressure), composition and
temperature are used to calculate the phase compositions, stored in an
auxiliary data class for each phase:

\begin{lstlisting}
  class EquationOfStateData {
    boolean present; // Phase present?

    double      V; // $V^\ell$
    Composition N; // $N_\nu^\ell$
    double     xi; // $\xi^\ell$

    double     dVdp; // $\left.\partial V^\ell/\partial p\right|_{T,N_\nu^\ell,\forall\nu}$
    double     dVdT; // $\left.\partial V^\ell/\partial T\right|_{p,N_\nu^\ell,\forall\nu}$
    double[]   dVdN; // $\left.\partial V^\ell/\partial N_\nu^\ell\right|_{p,T,N_\mu^\ell,\forall\mu\neq\nu}$

    double[]   dNdp; // $\left.\partial N_\nu^\ell/\partial p\right|_{p,T,N_\mu,\forall\mu}$
    double[]   dNdT; // $\left.\partial N_\nu^\ell/\partial T\right|_{p,T,N_\mu,\forall\mu}$
    double[][] dNdN; // $\left.\partial N_\mu^\ell/\partial N_\nu\right|_{p,T,N_\mu,\forall\mu\neq\nu}$

    double h; // $\rho^\ell h^\ell$
    double c; // $\left.\partial\left(\rho^\ell h^\ell\right)/\partial T\right|_{p,N_\nu,\forall\nu}$

    double mu; // $\mu^\ell$
  }
\end{lstlisting}

%======================================================================

% \csection{PVT calculations}

% The PVT calculations use a PVT class which stores the basic functions,
% provided on an input-file as tables. Obviously, only a single instance
% of the PVT class is needed.

% \begin{lstlisting}
%   class PVT {

%     // Saturated solubilities
%     PhaseData<Function2D> R;

%     // Volume factors
%     PhaseData<Function3D> B;

%     // Enthalpies
%     PhaseData<Function3D> h;

%     // Viscosities
%     PhaseData<Function3D> mu;

%     // PVT calculations for saturated conditions (free oil and gas)
%     PVTData calculateSaturated(
%        PhaseData<Double> p, // Phase pressures
%        double T,            // Temperature
%        PVTData data         // PVT data storage
%       );

%     // PVT calculations for undersaturated oil (no gas phase)
%     PVTData calculateUndersaturatedOil(
%        PhaseData<Double> p, // Phase pressures
%        double Ro,           // Solubility of the gas in the oil phase
%        double T,            // Temperature
%        PVTData data         // PVT data storage
%       );

%     // PVT calculations for undersaturated gas (no oil phase)
%     PVTData calculateUndersaturatedGas(
%        PhaseData<Double> p, // Phase pressures
%        double Rg,           // Solubility of the oil in the gas phase
%        double T,            // Temperature
%        PVTData data         // PVT data storage
%       );
%   }
% \end{lstlisting}

% The current phase state dictates which PVT calculation function to
% call.

% %----------------------------------------------------------------------

% \csubsection{PVT data}

% PVTData is a storage class with the following content:

% \begin{lstlisting}
%   class PVTData {
%     PhaseData<Solubility> solubility;
%     PhaseData<Density>    density;
%     PhaseData<Enthalpy>   enthalpy;
%     PhaseData<Viscosity>  viscosity;
%   }
% \end{lstlisting}

% It has four auxiliary classes which store phase specific
% data. Saturated solubilities and derivatives are:

% \begin{lstlisting}
%   class Solubility {
%     // Saturated solubility from table look-up, $R^\ell$
%     double Rsat;

%     // Derivatives of $R^\ell$
%     double dRdp; // $\partial R^\ell/\partial p^\ell$
%     double dRdT; // $\partial R^\ell/\partial T$
%   }
% \end{lstlisting}

% Density data and needed derivatives are stored as:

% \begin{lstlisting}
%   class Density {
%     // Volume factors $B^\ell$ and the inverse $b^\ell$
%     double B, b;

%     // Phase densities $\rho^\ell$
%     double rho;

%     // Partial derivatives of $B^\ell$
%     double dBdp; // $\partial B^\ell/\partial p^\ell$
%     double dBdR; // $\partial B^\ell/\partial R^\ell$
%     double dBdT; // $\partial B^\ell/\partial T$

%     // Derivatives of $b^\ell$
%     double dbdp; // $db^\ell/d p^\ell$
%     double dbdR; // $\partial b^\ell/\partial R^\ell$
%     double dbdT; // $db^\ell/d T$

%     // Density derivatives
%     double drhodp; // $d\rho^\ell/d p^\ell$
%     double drhodR; // $\partial\rho^\ell/\partial R^\ell$
%     double drhodT; // $d\rho^\ell/d T$
%   }
% \end{lstlisting}

% Enthalpy, internal energy, and their derivatives are:

% \begin{lstlisting}
%   class Enthalpy {
%     // Enthalpy and internal energy
%     double h, e;

%     // Partial enthalpy derivatives
%     double dhdp; // $\partial h^\ell/\partial p^\ell$
%     double dhdR; // $\partial h^\ell/\partial R^\ell$
%     double dhdT; // $\partial h^\ell/\partial T$

%     // Internal energy derivatives
%     double dedp; // $d e^\ell/d p^\ell$
%     double dedR; // $\partial e^\ell/\partial R^\ell$
%     double dedT; // $d e^\ell/d T$

%   }
% \end{lstlisting}

% Viscosity and $v^\ell=b^\ell/\mu^\ell$ with derivatives are:

% \begin{lstlisting}
%   class Viscosity {
%     // Viscosity and inverse volume factor over viscosity
%     double mu, v;

%     // Derivatives of $v^\ell$
%     double dvdp; // $d v^\ell/d p^\ell$
%     double dvdR; // $\partial v^\ell/\partial R^\ell$
%     double dvdT; // $d v^\ell/d T$
%   }
% \end{lstlisting}

%%% Local Variables: 
%%% mode: latex
%%% TeX-master: "im"
%%% End: 
