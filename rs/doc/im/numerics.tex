\chapter{Numerics}
\label{chap:numerics}

\minitoc

The flow simulator starts with \texttt{numerics.Main}. It reads in the
run specification, and creates one \texttt{SimulatorThread} for each
subdomain created by the mesh generator. The threads are then all
started and run concurrently for the duration of the simulation.

A simulator thread, when started, carries out these steps:
\begin{enumerate}
\item Reads in the subdomain mesh.
\item Creates a \texttt{field.Field}, which stores the primary
  variables and calculates the secondary variables.
\item Creates a \texttt{numerics.Discretisation} for flux
  discretisation, and a \texttt{numerics.TimeStepper} for reporting
  and administration.
\item Passes control to the timestepper, which uses the discretisation
  to update the field state.
\end{enumerate}

%======================================================================

\csection{Flowcharts}

The timestepper advances an initial field state forward to the desired
endtime, and outputs the state at the given report times. When
started, with its \texttt{doTimeStepping} method, it proceeds as
follows:
\begin{enumerate}
\item Report production statistics for all outlets.
\item Output field state, if desired.
\item Possible reduce the timestep $\Delta t$ to meet the next report
  time.
\item Advance the field state forward by $\Delta t$ using the
  \texttt{Discretisation} object:
  \begin{itemize}
  \item If a failure is detected, reduce the timestep, retract the
    field state, and retry.
  \end{itemize}
\item Get an adjusted timestep from the \texttt{Discretisation}.
\end{enumerate}

Similarly, \texttt{Discretisation} has a \texttt{solve} method for
determining the state at a new timestep:
\begin{enumerate}
\item Store the field state, for use by the timestepper in case of
  failures.
\item Solve the volume balance equation for the pressure
  (\texttt{PressureNewtonRaphson}):
  \begin{enumerate}
  \item\label{disc:1} Assemble the Darcy and Fourier flux contributions.
  \item Assemble the compressibility and source contributions.
  \item Solve the resulting linear system using \texttt{LinearSolver}.
  \item Update the pressure, and check for non-positive pressures.
  \item Calculate new secondary variables in the field.
  \item Check for convergence, and if not, continue from step
    \ref{disc:1}.
  \end{enumerate}
\item If a thermal run, solve the energy conservation law for the
  temperature (\texttt{TemperatureNewtonRaphson}):
  \begin{itemize}
  \item Assemble the advective Darcy fluxes explicitly.
  \item Assemble the conductive Fourier fluxes implicitly.
  \item Assemble in the capacity and source terms.
  \item Solve the resulting system by the \texttt{LinearSolver}.
  \item Update the temperature, and ensure that the temperature stays
    positive.
  \end{itemize}
\item Solve explicitly for the molar masses
  (\texttt{MolarMassExplicit}):
  \begin{itemize}
  \item Calculate flux contributions.
  \item Calculate contributions from mass sources.
  \item Update, and ensure that the mole numbers stay non-negative.
  \end{itemize}
\end{enumerate}

%----------------------------------------------------------------------

\csection{Linear solver}

A parallel linear solver is used for the solution of the linear
systems for pressure and temperature. It uses the
MTJ\footnote{\texttt{http://rs.cipr.uib.no/mtj}} library, and in
particular its distributed memory matrices and solvers.

%%% Local Variables: 
%%% mode: latex
%%% TeX-master: "im"
%%% End: 
