\chapter{Field}
\label{chap:field}

\minitoc

The \texttt{field} package is primarily an interface between the
numerical flow simulator and the \texttt{fluid} and \texttt{rock}
packages. \texttt{field.Field} is as follows:
\begin{lstlisting}
  class Field {

    // Gets the control volume for the given element
    CV getControlVolume(Element el);

    // Gets the control surface for the given connection
    CS getControlSurface(Connection c);

    // Lists the mass and heat sources
    Iterable<Source> sources();

    // Calculates secondary variables for all CV's and CS's
    void calculateSecondaries();

    // Gets the current time in seconds
    double getTime();

    // $t = t + \Delta t$
    void changeTime(double dt);

    // Rewinds the field state to the given time with the given state
    void retract(double t, double[] p, double[] T, Composition[] N);
  }
\end{lstlisting}

The class \texttt{CV} stores all primary and secondary variables in an
element, and calculates all secondary quantities from an equation of
state and the associated \texttt{RockFluid} object.

\texttt{CS} does likewise on a connection. It determines the upstream
directions and Darcy fluxes for all the phases.

A \texttt{Field} also calculates an initial fluid state using the
\texttt{InitialValues} class. The exact procedure is given in the
Technical Description.

%%% Local Variables: 
%%% mode: latex
%%% TeX-master: "im"
%%% End: 
