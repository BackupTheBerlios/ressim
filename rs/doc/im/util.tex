\chapter{Utilities}

\minitoc

The utilities package contains functionality which is used by all the
other packages, such as parsing of configuration files, using
tabulated data, and reporting information back to the user.

%======================================================================

\csection{Configuration files}

%======================================================================

\csection{Table look-up}

%======================================================================

\csection{Logging}

Simulator output is of two types: messages written to the console
during execution, and field states written to separate files. The
former is divided into two subtypes:
\begin{description}
\item[$\lstinline{System.out}$] Progress information (from
  timestepping, Newton iterations, and linear iterations), and time
  usage statistics.
\item[$\lstinline{System.err}$] Internal details from the simulator
  initialization and possibly error messages. In the latter case, the
  simulator must stop.
\end{description}

%======================================================================

\csection{Time measurements}

The time usage of the simulator may be performed using a clock:
\begin{lstlisting}
  class Clock {

    // Starts a named timer
    static void start(String timer);

    // Stops a named timer
    static void stop(String timer);

    // Creates a printable report over all the timers
    static String report();
  }
\end{lstlisting}

The timestepper outputs time usage at each sucessful timestep.

%%% Local Variables: 
%%% mode: latex
%%% TeX-master: "im"
%%% End: 
