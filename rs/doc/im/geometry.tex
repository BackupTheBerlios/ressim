\chapter{Geometry}
\label{chap:geometry}

\minitoc

The geometry package contains the general mesh structure for an
arbitrary unstructured mesh, and some transmissibility calculation
methods. Its namespace is \texttt{no.uib.cipr.rs.geometry}.

%======================================================================

\csection{Mesh structure}

The mesh by the class \texttt{Mesh}, and its interface is as follows:

\begin{lstlisting}
  class Mesh {

    // Mesh constructor
    Mesh(Geometry, Topology, Rock[]);

    // Corner points
    List<Point> points();

    // Interfaces
    List<Interface> interfaces();

    // Elements
    List<Element> elements();

    // Neighbour (geometrically) connections
    List<NeighbourConnection> neighbourConnections();

    // Non-neighbour connections
    List<Connection> nonNeighbourConnections();

    // All the connections (neighbours and non-neighbours)
    List<Connection> connections();
  }
\end{lstlisting}

A mesh is constructed from a \texttt{Geometry} and a
\texttt{Topology}, along with petrophysical rock data for each
element. The mesh generators are responsible for creating these
objects.

%----------------------------------------------------------------------

\csubsection{Geometrical constructs}

An \texttt{Element} is a general polyhedral entiety, with references
to its cornerpoints and interfaces:
\begin{lstlisting}
  class Element {

    // Linear index
    int getIndex();

    // Petrophysical data
    Rock getRock();

    // Volume in cubic meters
    double getVolume();

    // Coordinate of the center
    Point3D getCenter();

    // Corner points of this element
    List<Point> points();

    // Interfaces of this element
    List<Interface> interfaces();

    // Associated non-neighbour connections
    List<Connection> nonNeighbourConnections();
  }
\end{lstlisting}

Next, an interface is a planar side of an element.
\begin{lstlisting}
  class Interface {

    // Linear index
    int getIndex();

    // Surface area, in square meters
    double getArea();

    // Center point
    Point3D getCenter();

    // Outward normal vector
    Vector3D getNormal();

    // Associated corner points
    List<Point> points();

    // The single associated element
    Element element();

    // Boundary interfaces do not have a connection
    boolean isBoundary();

    // Internal interfaces have neighbour connections
    NeighbourConnection connection();
  }
\end{lstlisting}
Note that neighbouring interfaces are stored distinctly, as they have
different outward normal directions.

The last geometrical construct is the \texttt{Point}:
\begin{lstlisting}
  class Point {

    // Linear index
    int getIndex();

    // Geometrical coordinate
    Point3D getCoordinate();

    // Interfaces refering to this point
    List<Interface> interfaces();

    // Elements refering to this point
    List<Element> elements();
  }
\end{lstlisting}

%----------------------------------------------------------------------

\csubsection{Topological constructs}

A \texttt{Connection} connects two elements for mass and heat
flow. The connected elements need not be geometrical neighbours.
\begin{lstlisting}
  class Connection {

    // Linear index
    int getIndex();

    // Element on the here side
    Element hereElement();

    // Element on the there side
    Element thereElement();

    // Transmissibility multiplier for Darcy flow
    double getMultiplier();

    // Darcy transmissibilities for mass flow
    Transmissibility[] getDarcyTransmissibilities();

    // Fourier transmissibilities for heat flow
    Transmissibility[] getFourierTransmissibilities();
  }
\end{lstlisting}
Note that the multiplier will be factored into the Darcy
transmissibilities automatically by the class constructor.

A neighbouring connection is a subclass, which also links adjacent
interfaces:
\begin{lstlisting}
  class NeighbourConnection extends Connection {

    // Interface on the here side
    Interface hereInterface();

    // Interface on the there side
    Interface thereInterface();
  }
\end{lstlisting}

%======================================================================

\csection{Transmissibility calculations}

Transmissibilities are found for both the absolute permeability (Darcy
flow) and the rock heat conductivity (Fourier flow). Since the method
of calculation is the same, the tensors are abstracted away with the
following interface:
\begin{lstlisting}
  interface Conductivity {
    Tensor3D getConductivity(Element el);
  }
\end{lstlisting}

A \texttt{TransmissibilityComputer} uses this interface to calculate
transmissibilities for each neighbouring grid connection.
Transmissibilities for non-neighbouring connections must be set
externally, as they cannot be found automatically. A transmissibility
is then simply
\begin{lstlisting}
  class Transmissibility {

    // Transmissibility factor
    double getTransmissibility();

    // Associated element
    Element getElement();
  }
\end{lstlisting}

%%% Local Variables: 
%%% mode: latex
%%% TeX-master: "im"
%%% End: 
