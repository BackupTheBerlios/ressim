\chapter{Conservation laws}

\minitoc

The fluid system consists of three phases indexed with the superscript
$\ell$: water ($\ell=w$), oil ($\ell=o$) and gas ($\ell=g$). Fluid
pressure and temperature varies, and the fluids consists of components
indexed by the subscript $\nu$.

%======================================================================

\csection{Volume balance}

Following \cite{watts1986}, a volume balance equation is used for
determining the fluid pressure and velocities. At all times, the fluid
fills the pore volumes, and consequently the residual volume
\begin{equation}
  R = V^p - V^f = \phi V - \sum_\ell V^\ell = 0,
\end{equation}
where $V^p=\phi V$ is the pore volume, $\phi$ is the porosity, $V$ is
the bulk volume, $V^f=\sum_\ell V^\ell$ is the fluid volume, and
$V^\ell$ is the phase volume. As the residual volume $R$ is zero at
all times, its timederivative must also be zero.  Assuming that the
residual volume is a function of the water pressure $p^w$, temperature
$T$, and component masses $N_\nu$, the timederivative may be expanded
into
\begin{equation}
  \frac{\partial R}{\partial t} =
  \frac{\partial R}{\partial p^w} \frac{\partial p^w}{\partial t} +
  \frac{\partial R}{\partial T} \frac{\partial T}{\partial t} +
  \sum_\nu \frac{\partial R}{\partial N_\nu}
  \frac{\partial N_\nu}{\partial t} = 0,
\end{equation}
in which
\begin{eqnarray}
  \frac{\partial R}{\partial p^w} & = &
  \frac{\partial\phi}{\partial p} V
  -\sum_\ell \frac{\partial V^\ell}{\partial p^w}, \\
  \frac{\partial R}{\partial T} & = &
  -\sum_\ell \frac{\partial V^\ell}{\partial T}, \\
  \frac{\partial R}{\partial N_\nu} & = &
  -\sum_\ell \frac{\partial V^\ell}{\partial N_\nu}.
\end{eqnarray}
The residual volume at the next time is then given by the Taylor
expansion
\begin{equation}
  R^{n+1} \approx R^n + \frac{\partial R}{\partial t} \Delta t.
\end{equation}
Setting $R^{n+1}=0$, the equation is divided by $\Delta t$, yielding
\begin{equation}
  \frac{R^n}{\Delta t} + \frac{\partial R}{\partial t} = 0.
\end{equation}

%----------------------------------------------------------------------

\subsection{Compaction}

Porosity will in general depend on the fluid pressure, $p=p^w$, and
this dependency is taken into account by the rock compressibility
$c^r$:
\begin{equation}
  c^r = \frac{1}{\phi} \frac{\partial\phi}{\partial p},
  \quad c^r\geq 0.
\end{equation}
Assuming a constant compressibility, this equation is integrated
yielding
\begin{eqnarray}
  \phi(p) & = & \phi_0 \exp\left( c^r \Delta p \right), \\
  & \approx & \phi_0 \left( 1 + c^r \Delta p +
    \frac{1}{2} \left(c^r\right)^2 \Delta p^2 \right),
  \quad \Delta p = p-p_0.
\end{eqnarray}
The reference porosity $\phi_0$ is given at the initial pressure
$p_0$, which may be taken to the initial water phase pressure.

%----------------------------------------------------------------------

\subsection{Initialisation}

The fluid system is initialised such that the residual volume is zero
everywhere, and that the fluids are in a hydrostatic pressure
distribution. Given a pressure at a datum depth, $p(z_D)=p_D$, a
temperature as a function of depth, $T(z)$, and a overall composition
also as a function of depth, $C_\nu(z)$, the initial pressures in all
gridblocks is set to the datum pressure, and an initial amount of
total moles in each gridblock is (arbitrarily) set. Then the iteration
proceeds:
\begin{enumerate}
\item Solve the volume balance for the total amount of mass, $N$:
  \[
  \left(\sum_\nu \frac{\partial R}{\partial N_\nu} C_\nu\right)
  \Delta N = -R.
  \]
\item Calculate new phase states by an equation of state.
\item Set new gridblock pressure by a hydrostatic relation:
  \[
  p = p_D - g \sum_\ell \rho^\ell S^\ell \left(
    z - z_D
  \right).
  \]
\end{enumerate}
The iteration terminates once $R=0$.

%======================================================================

\csection{Conservation of mass}

The compositional mass conservation law states
\cite{allen1988,peaceman1978}
\begin{equation}
  \frac{\partial N_\nu}{\partial t} + \int_{\partial\Omega}
  \sum_\ell \xi^\ell C_\nu^\ell \vec u^\ell\cdot\vec n\, dS = q_\nu,
\end{equation}
where $N_\nu$ is the component mass, $\xi^\ell$ is the phase molar
density, $C_\nu^\ell$ is the mole fraction of component $\nu$ in phase
$\ell$, $q_\nu$ is the mass source, and $\vec u^\ell$ is the phase
Darcy velocity:
\begin{equation}
  \vec u^\ell = -\lambda^\ell \tensor{K}\nabla\Psi^\ell.
\end{equation}
Herein, $\lambda^\ell=k_r^\ell/\mu^\ell$ is the phase mobility,
$k_r^\ell$ is its relative permeability, $\mu^\ell$ is the phase
viscosity, $\tensor K$ is the absolute permeability, and
$\nabla\Psi^\ell$ is the phase potential gradient:
\begin{equation}
  \nabla\Psi^\ell = \nabla p^\ell + \rho^\ell \nabla b,
\end{equation}
in which $p^\ell$ is the phase pressure, $\rho^\ell$ the mass
density, and $b$ is the body force, given by
\begin{equation}
  \nabla b = g \nabla z + \frac{\omega^2}{2}
   \nabla\|\vec a_{x,y} - \vec (x,y,0)\|_2^2.
\end{equation}
$g$ is the gravitational acceleration
constant, $z$ is the coordinate along the $z$-axis (pointing upwards),
$\omega$ is the centrifuge rotational speed, $\vec a$ is the centrifuge
center in the $(x,y)$-plane, and the norm is the Euclidian distance from
the center to the current coordinate, projected into the $(x,y)$-plane.

%----------------------------------------------------------------------

\subsection{Capillary pressures and relative permeabilities}

Phase saturations are defined by
\begin{equation}
  S^\ell = \frac{V^\ell}{\sum_\ell V^\ell}.
\end{equation}
Notice that $\sum_\ell S^\ell=1$, such that $S^g=1-S^o-S^w$. The phase
pressures are related by the capillary pressures:
\begin{eqnarray}
  p^o & = & p^w + p_c^{ow}\left(S^w\right), \\
  p^g & = & p^o + p_c^{go}\left(S^g\right) =
  p^w + p_c^{ow}\left(S^w\right) + p_c^{go}\left(S^g\right).
\end{eqnarray}
Relative permeabilities are assumed to have the functional dependency
\begin{equation}
  k_r^\ell = k_r^\ell\left( S^w, S^g \right).
\end{equation}

%----------------------------------------------------------------------

\subsection{Mass sources}

The mass source, $q_\nu$, may either be given explicitly for each
component, or a total sink, $q_T<0$, may be given. Then the individual
$q_\nu$ are:
\begin{eqnarray}
  q_\nu & = & \sum_\ell C_\nu^\ell q^\ell, \\
  q^\ell & = & f^\ell q_T, \\
  f^\ell & = & \frac{\lambda^\ell}{\sum_\ell \lambda^\ell}.
\end{eqnarray}

%======================================================================

\csection{Conservation of energy}

The general energy conservation law states \cite{allen1988,welty1984}
\begin{equation}
  \frac{\partial}{\partial t}
  \int_\Omega\left(
    (1-\phi)\rho^r e^r +
    \phi \sum_\ell S^\ell \rho^\ell e^\ell
  \right)\, dV +
  \int_{\partial\Omega} \left(
    \sum_\ell \rho^\ell h^\ell \vec u^\ell -
    \kappa\nabla T
  \right)\cdot\vec n\, dS = q^e.
\end{equation}
It includes the energy stored in the rock phase and heat conduction in
the bulk matrix, in addition to fluid energy and transport. $e^\ell$
is the specific internal energy of phase $\ell$, $h^\ell$ is the
enthalpy, and $\kappa$ is the rock heat conductivity tensor. The
system temperature is $T$, and Fourier's law has been used to describe
the conductive heat flow. $q^e$ is the energy source.

The enthalpy contains both the internal energy and the compressive
heating, and neglecting the latter, $h^\ell=e^\ell$. Furthermore,
assuming that only the enthalpy density $\rho^\ell h^\ell$ exhibits
material change as the temperature is varied, the energy conservation
law becomes
\begin{equation}
  \int_\Omega\left(
    (1-\phi) \frac{\partial\left(\rho^r h^r\right)}{\partial T} +
    \phi\sum_\ell
    S^\ell \frac{\partial\left(\rho^\ell h^\ell\right)}{\partial T}
  \right)\frac{\partial T}{\partial t}\, dV +
  \int_{\partial\Omega} \left(
    \sum_\ell \rho^\ell h^\ell \vec u^\ell -
    \kappa\nabla T
  \right)\cdot\vec n\, dS = q^e.
\end{equation}
Assuming piecewise constant properties yields the final form of the
energy conservation law:
\begin{equation}
  \underbrace{\left(
      V^r \frac{\partial\left(\rho^r h^r\right)}{\partial T} +
      \sum_\ell
      V^\ell \frac{\partial\left(\rho^\ell h^\ell\right)}{\partial T}
    \right)}_{\delta^T}\frac{\partial T}{\partial t} +
  \int_{\partial\Omega} \left(
    \sum_\ell \rho^\ell h^\ell \vec u^\ell -
    \kappa\nabla T
  \right)\cdot\vec n\, dS = q^e.
\end{equation}
The temperature derivative of $\rho^\ell h^\ell$ is referred to as the
phase heat capacity, and it must be positive.

%%% Local Variables: 
%%% mode: latex
%%% TeX-master: "td"
%%% End: 
