\chapter{Control volume discretisation}

\minitoc

A control volume finite difference method is used for calculating the
Darcy and Fourier fluxes between the gridcells. The overall procedure
is to solve implicitly for the pressure using the volume balance
equation, and then explicitly calculate masses and temperature. Hence
the solution procedure for a given timestep is:
\begin{enumerate}
\item Solve the volume balance equation for the pressure: $p^w$.
\item Calculate volumetric Darcy phase fluxes: $D^\ell$.
\item Solve for temperature: $T$.
\item Solve for the masses: $N_\nu$.
\item Calculate a new thermodynamical equilibrium.
\end{enumerate}

%======================================================================

\csection{Flux discretisations}

Properties inside of a control volume are taken to be spatially
constant, and as such it is only the flux discretisation which needs
closer attention. The surface of the control volume $\Omega_i$ is
denoted $\partial\Omega_i$, and this surface is further subdivided
into planar faces
\begin{equation}
  \partial\Omega_i = \bigcup_{k=1}^{n_i} \partial\Omega_{i,k}.
\end{equation}
Furthermore, let $\mathcal{M}_{i,k}^\tensor{H}$ contain the set of
control volume indices contributing to the flux over
$\partial\Omega_{i,k}$. Then a flux integral can be approximated with
\begin{eqnarray}
  -\int_{\partial\Omega_i} \lambda\tensor{H}\nabla\Psi\cdot\vec n\, dS
  & = & -\sum_{k=1}^{n_i} \int_{\partial\Omega_{i,k}}
  \lambda\tensor{H}\nabla\Psi\cdot\vec n_{i,k}\, dS \\
  & \approx & \sum_{k=1}^{n_i} \lambda_{i,k}
  \sum_{j\in\mathcal{M}_{i,k}^\tensor{H}} t_{ij} \Psi_j.
\end{eqnarray}
$t_{ij}$ are the geometrical coupling coefficients, or simply the
transmissibilities. They are only dependent on the medium conductivity
$\tensor{H}$ and the geometry, and therefore may be computed as a
preprocessing step. See \cite{aavatsmarkell} for details on their
computation.

The mobility $\lambda$ is evaluated upstream on
$\partial\Omega_{i,k}$, yielding $\lambda_{i,k}$. The upstream
direction is given by the sign of the summation
\begin{equation}
  \sum_{j\in\mathcal{M}_{i,k}^\tensor{H}} t_{ij} \Psi_j.
\end{equation}
If is is positive, the flow is outwards and control volume $i$ is
chosen. Otherwise, the neighbour value is used instead.

%----------------------------------------------------------------------

\csubsection{Darcy flux}

A Darcy flux integral is approximated as
\begin{eqnarray}
  D_{i,k}^\ell = \int_{\partial\Omega_{i,k}} \vec u^\ell\cdot\vec n\, dS
  & \approx &
  - \lambda_{i,k}^\ell\int_{\partial\Omega_{i,k}}
  \tensor{K}\nabla\Psi^\ell\cdot\vec n\, dS \\
  & \approx &
  \lambda_{i,k}^\ell
  \sum_{j\in\mathcal{M}_{i,k}^\tensor{K}} t_{ij} \Psi_j^\ell.
\end{eqnarray}
The cell centred phase potential is given by
\begin{equation}
  \Psi_j^\ell = p_j^\ell + \bar\rho^\ell_{i,k} b_j,
\end{equation}
where the face saturation averaged phase mass density is
\begin{equation}
  \bar\rho^\ell_{i,k} =
  \frac{S_i^\ell \rho_i^\ell + S_k^\ell \rho_k^\ell}{S_i^\ell + S_k^\ell},
\end{equation}
and the body force is
\begin{equation}
	b_j = g z_j + \frac{\omega^2}{2}
   \|\vec a_{x,y} - \vec (x_j,y_j,0)\|_2^2.
\end{equation}

The volume balance formulation requires the pressure derivative of the
Darcy flux, which is approximated as:
\begin{equation}
  \frac{\partial}{\partial p_l^o} D_{i,k}^\ell \approx
  \lambda_{i,k}^\ell
  \sum_{j\in\mathcal{M}_{i,k}^\tensor{K}} t_{ij} \delta_{jl}.
\end{equation}

%----------------------------------------------------------------------

\subsubsection{Mass flux}

The mass flux of component $\nu$ is
\begin{equation}
  \sum_\ell \int_{\partial\Omega_i} \xi^\ell C_\nu^\ell \vec
  u^\ell\cdot\vec n\, dS \approx
  \sum_{k=1}^{n_i} \sum_\ell
  \left(\xi^\ell C_\nu^\ell\right)_{i,k}
  D_{i,k}^\ell.
\end{equation}
The pressure derivative of the mass flux only considers the variation
of the Darcy flux, and not $\xi^\ell C_\nu^\ell$.

%----------------------------------------------------------------------

\subsubsection{Energy flux}

The advective energy flux is approximated by
\begin{equation}
  \sum_\ell \int_{\partial\Omega_i} \rho^\ell h^\ell \vec
  u^\ell\cdot\vec n\, dS \approx
  \sum_{k=1}^{n_i} \sum_\ell
  \left(\rho^\ell h^\ell\right)_{i,k}
  D_{i,k}^\ell.
\end{equation}
As for the mass flux, it is only $D_{i,k}^\ell$ which varies with the
pressure.

%----------------------------------------------------------------------

\csubsection{Fourier flux}

The Fourier (conductive) energy flux is
\begin{equation}
  F_{i,k} = -\int_{\partial\Omega_i} \kappa\nabla T\cdot\vec n\, dS
  \approx
  \sum_{k=1}^{n_i} \sum_{j\in\mathcal{M}_{i,k}^\kappa} k_{ij} T_j.
\end{equation}
Its pressure derivative is zero, while its temperature derivative is
\begin{equation}
  \frac{\partial}{\partial T_l} F_{i,k} \approx
  \sum_{k=1}^{n_i} \sum_{j\in\mathcal{M}_{i,k}^\kappa} k_{ij} \delta_{jl}.
\end{equation}

%======================================================================

\csection{Volume balance discretisation}

The volume balance equation is solved for the pressure, and the
residual equation in each gridblock is
\begin{equation}
  r_i = \frac{R^{n+1}_i}{\Delta t} + \frac{\partial R_i}{\partial p^o}
  \frac{p_i^{n+1,w} - p_i^{n,w}}{\Delta t} +
  \frac{\partial R_i}{\partial T}\frac{\partial T_i}{\partial t} +
  \sum_\nu \frac{\partial R_i}{\partial N_\nu}
  \frac{\partial N_{i,\nu}}{\partial t}.
\end{equation}
The temperature timederivative is
\begin{equation}
  \frac{\partial T_i}{\partial t} \approx
  \frac{q_i^e}{\delta_i^T}-\frac{1}{\delta_i^T}
  \sum_{k=1}^{n_i} \left(\sum_\ell
    \left(\rho^\ell h^\ell\right)_{i,k}
    D_{i,k}^\ell + F_{i,k}
  \right).
\end{equation}
The mass timederivative is
\begin{equation}
  \frac{\partial N_{i,\nu}}{\partial t} \approx
  q_{i,\nu}-\sum_{k=1}^{n_i} \sum_\ell
  \left(\xi^\ell C_\nu^\ell\right)_{i,k} D_{i,k}^\ell.
\end{equation}
The residual $r_i$ is a nonlinear function of the water pressure, so a
Newton-Raphson iteration is appropriate. Its pressure derivative is
approximated with
\begin{equation}
  \frac{\partial}{\partial p_l^w} r_i \approx
  \frac{\partial R_i}{\partial p^w}
  \frac{\delta_{il}}{\Delta t} +
  \frac{\partial R_i}{\partial T}
  \frac{\partial}{\partial p_l^w} \frac{\partial T_i}{\partial t} +
  \sum_\nu \frac{\partial R_i}{\partial N_\nu}
  \frac{\partial}{\partial p_l^w} \frac{\partial N_{i,\nu}}{\partial t},
\end{equation}
in which $\delta_{il}$ is the Dirac delta function. The second order
temperature derivative is approximated
\begin{equation}
  \frac{\partial}{\partial p_l^w} \frac{\partial T_i}{\partial t} \approx
  -\frac{1}{\delta_i^T}
  \sum_{k=1}^{n_i} \sum_\ell
  \left(\rho^\ell h^\ell\right)_{i,k}
  \frac{\partial D_{i,k}^\ell}{\partial p_l^w},
\end{equation}
and likewise the second order mass derivative is
\begin{equation}
  \frac{\partial}{\partial p_l^w}
  \frac{\partial N_{i,\nu}}{\partial t} \approx
  -\sum_{k=1}^{n_i} \sum_\ell
  \left(\xi^\ell C_\nu^\ell\right)_{i,k}
  \frac{\partial D_{i,k}^\ell}{\partial p_l^w}.
\end{equation}

The Newton-Raphson iteration is now to solve the linear matrix system
\begin{equation}
  \sum_l \frac{\partial r_i}{\partial p_l^w} \Delta p_l^w =
  -r_i,\quad\forall i.
\end{equation}

%======================================================================

\csection{Mass discretisation}

Using an explicit forward Euler time discretisation, the mass
equations become
\begin{equation}
  N_{i,\nu}^{n+1} = N_{i,\nu}^n +
  \Delta t \left(
    q_{i,\nu} -
    \sum_{k=1}^{n_i} \sum_\ell
    \left(\xi^\ell C_\nu^\ell\right)_{i,k} D_{i,k}^\ell
  \right).
\end{equation}

%======================================================================

\csection{Energy discretisation}

The discretisation of the energy equation treats the Fourier flux
implicitly while the Darcy flux is taken explicitly. Its block
residual is
\begin{equation}
  r_i = \delta_i^T \frac{T_i^{n+1} - T_i^n}{\Delta t} +
  \sum_{k=1}^{n_i} \left(\sum_\ell
    \left(\rho^\ell h^\ell\right)_{i,k}
    D_{i,k}^\ell + F_{i,k}
  \right) - q_i^e.
\end{equation}
The Jacobian matrix is formed from
\begin{equation}
  \frac{\partial r_i}{\partial T_l} \approx
  \delta_i^T\frac{\delta_{il}}{\Delta t} +
  \frac{\partial F_{i,k}}{\partial T_l}.
\end{equation}

%%% Local Variables: 
%%% mode: latex
%%% TeX-master: "td"
%%% End: 
