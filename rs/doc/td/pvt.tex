\chapter{PVT properties}
\label{chap:pvt}

\minitoc

The PVT properties of the system are fluid properties which are
dependent on the pressure, temperature and masses. These are volumes,
densities, phase compositions, enthalpies and viscosities. Phase
volume derivatives will also be calculated, since these are required
for the volume balance equation.

Throughout this chapter, the pressure is always the water pressure,
that is $p=p^w$.

Total and phase mole fractions are defined by
\begin{eqnarray}
  C_\nu & = & \frac{N_\nu}{\sum_\nu N_\nu}, \\
  C_\nu^\ell & = & \frac{N_\nu^\ell}{\sum_\nu N_\nu^\ell}.
\end{eqnarray}
The phase molar and mass densities are
\begin{eqnarray}
  \xi^\ell & = & \frac{\sum_\nu N_\nu^\ell}{V^\ell}, \\
  \rho^\ell & = & \frac{\sum_\nu M_\nu N_\nu^\ell}{V^\ell},
\end{eqnarray}
where $M_\nu$ is the molecular weight of the component $\nu$.

%======================================================================

\csection{Black-oil}

The black-oil model has three components: water ($\nu=w$), oil
($\nu=o$), and gas ($\nu=g$). The water component is restricted to the
water phase, while the oil and gas components may exists in both the
water and oil phases. If the fluid system has more components, they
must be grouped into these component sets. Let $\nu\in\mathcal{G}$ be
gas components and $\nu\in\mathcal{O}$ be oil components, such that
these two sets are disjoint. The reduction is now:
\begin{eqnarray}
  N_g & = & \sum_{\nu\in\mathcal{G}} N_\nu, \\
  N_o & = & \sum_{\nu\in\mathcal{O}} N_\nu, \\
  N_\text{HC} & = & N_g + N_o.
\end{eqnarray}

%----------------------------------------------------------------------

\subsection{Phase split}

Let the liquid and vapor hydrocarbon fractions be
\begin{equation}
  L=\frac{N^o}{N_\text{HC}},\quad V=\frac{N^g}{N_\text{HC}},
\end{equation}
and notice that $V=1-L$. Then clearly
\begin{eqnarray}
  N_o^o & = & N^o C_o^o = L N_\text{HC}\left(1-C_g^o\right), \\
  N_g^o & = & N^o C_g^o = L N_\text{HC} C_g^o, \\
  N_o^g & = & N^g C_o^g = (1-L) N_\text{HC} C_o^g, \\
  N_g^g & = & N^g C_g^g = (1-L) N_\text{HC}\left(1-C_o^g\right).
\end{eqnarray}
The phase split thus requires $L$ and the two fractions $C_g^o$ and
$C_o^g$. To this end, for saturated fluids, the black-oil model uses a
dew-point curve
\begin{equation}
  C_{o,D}^g\left(p,T\right) = \frac{N_o^g}{N^g},
\end{equation}
and a bubble-point curve
\begin{equation}
  C_{g,B}^o\left(p,T\right) = \frac{N_g^o}{N^o}.
\end{equation}
Furthermore, define the overall amount of the oil and gas components
in the hydrocarbon mixture as
\begin{equation}
  Z_o = \frac{N_o}{N_\text{HC}}, \quad
  Z_g = \frac{N_g}{N_\text{HC}}.
\end{equation}

\newpage

\begin{figure}
  \begin{center}
    \begin{picture}(200,50)
      \put(0,25){\vector(1,0){200}}
      \put(210,25){\makebox(0,0){$Z_o$}}

      \put(20,32.5){\makebox(0,0){Gas}}
      \put(35,20){\line(0,1){10}}
      \put(35,10){\makebox(0,0){$C_{o,D}^g$}}

      \put(180,32.5){\makebox(0,0){Oil}}
      \put(165,20){\line(0,1){10}}
      \put(170,10){\makebox(0,0){$C_{o,B}^o$}}

      \put(100,32.5){\makebox(0,0){Gas \& Oil}}

      \put(140,25){\circle*{2.5}}
      \put(38,20){$\underbrace{\makebox(100,0){}}_{d_g}$}
      \put(143,20){$\underbrace{\makebox(20,0){}}_{d_o}$}
    \end{picture}
  \end{center}
  \caption{Binary phase diagram at fixed pressure and temperature.}
  \label{fig:binary}
\end{figure}

%......................................................................

\subsubsection{Single-phase oil}

The oil phase is undersaturated if the amount of gas is below the
bubble-point:
\begin{equation}
  Z_g \leq C_{g,B}^o\quad\Leftrightarrow\quad
  Z_o \geq C_{o,B}^o,
\end{equation}
where $C_{o,B}^o=1-C_{g,B}^o$. See Figure~\ref{fig:binary}. Then
$L=1$, $C_g^o=Z_g$ and $C_o^g=0$.

%......................................................................

\subsubsection{Single-phase gas}

Conversely, the gas phase is undersaturated if the amount of oil in it
is below the dew-point:
\begin{equation}
  Z_o \leq C_{o,D}^g\quad\Leftrightarrow\quad
  Z_g \geq C_{g,D}^g,
\end{equation}
where $C_{g,D}^g=1-C_{o,D}^g$. Now $L=0$, $C_o^g=Z_o$ and $C_g^o=0$.

%......................................................................

\subsubsection{Two-phase oil/gas}

If $Z_o$ lies between the dew- and bubble-points, two phases may
coexist. Referring again to Figure~\ref{fig:binary}, the liquid
hydrocarbon fraction is seen to be
\begin{equation}
  L = \frac{d_g}{d_g + d_o} =
  \frac{Z_o - C_{o,D}^g}{
    \left(Z_o - C_{o,D}^g\right)+
    \left(C_{o,B}^o - Z_o\right)} =
  \frac{Z_o - C_{o,D}^g}{1 - C_{o,D}^g - C_{g,B}^o}.
\end{equation}
Otherwise, since both phases are saturated, their composition is set
equal to the composition at the bubble- and dew-points. Hence,
$C_g^o=C_{g,B}^o$ and $C_o^g=C_{o,D}^g$.

%----------------------------------------------------------------------

\csubsection{Phase properties and derivatives}

The molar mass densities are the given functions
\begin{eqnarray}
  \xi^o & = & \xi^o\left(p,C_g^o,T\right), \\
  \xi^g & = & \xi^g\left(p,C_o^g,T\right).
\end{eqnarray}
Phase volumes and their derivatives are dependent on the actual phase
state.

%......................................................................

\subsubsection{Single-phase oil}

The oil volume is
\begin{equation}
  V^o = \frac{N_o + N_g}{\xi^o}.
\end{equation}
Its derivatives are
\begin{eqnarray}
  \frac{\partial V^o}{\partial p} & = &
  -\frac{V^o}{\xi^o} \frac{\partial \xi^o}{\partial p}, \\
  \frac{\partial V^o}{\partial T} & = &
  -\frac{V^o}{\xi^o} \frac{\partial \xi^o}{\partial T}, \\
  \frac{\partial V^o}{\partial N_o} & = &
  \frac{1}{\xi^o} -
  \frac{V^o}{\xi^o} \frac{\partial \xi^o}{\partial C_g^o}
  \frac{\partial C_g^o}{\partial N_o}, \\
  \frac{\partial V^o}{\partial N_g} & = &
  \frac{1}{\xi^o} -
  \frac{V^o}{\xi^o} \frac{\partial \xi^o}{\partial C_g^o}
  \frac{\partial C_g^o}{\partial N_g}.
\end{eqnarray}
As $C_g^o=N_g/(N_g+N_o)$, its derivatives are
\begin{eqnarray}
  \frac{\partial C_g^o}{\partial N_o} & = &
  -\frac{C_g^o}{N_\text{HC}}, \\
  \frac{\partial C_g^o}{\partial N_g} & = &
  \frac{1-C_g^o}{N_\text{HC}}.
\end{eqnarray}

%......................................................................

\subsubsection{Single-phase gas}

The gas volume is
\begin{equation}
  V^g = \frac{N_o + N_g}{\xi^g}.
\end{equation}
Its derivatives are
\begin{eqnarray}
  \frac{\partial V^g}{\partial p} & = &
  -\frac{V^g}{\xi^g} \frac{\partial \xi^g}{\partial p}, \\
  \frac{\partial V^g}{\partial T} & = &
  -\frac{V^g}{\xi^g} \frac{\partial \xi^g}{\partial T}, \\
  \frac{\partial V^g}{\partial N_o} & = &
  \frac{1}{\xi^g} -
  \frac{V^g}{\xi^g} \frac{\partial \xi^g}{\partial C_o^g}
  \frac{\partial C_o^g}{\partial N_o}, \\
  \frac{\partial V^g}{\partial N_g} & = &
  \frac{1}{\xi^g} -
  \frac{V^g}{\xi^g} \frac{\partial \xi^g}{\partial C_o^g}
  \frac{\partial C_o^g}{\partial N_g}.
\end{eqnarray}
Since $C_o^g=N_o/(N_g+N_o)$, its derivatives are
\begin{eqnarray}
  \frac{\partial C_o^g}{\partial N_o} & = &
  \frac{1-C_o^g}{N_\text{HC}}, \\
  \frac{\partial C_o^g}{\partial N_g} & = &
  -\frac{C_o^g}{N_\text{HC}}.
\end{eqnarray}

%......................................................................

\subsubsection{Two-phase oil/gas}

The oil and gas volumes are
\begin{eqnarray}
  V^o & = & \frac{N_o^o + N_g^o}{\xi^o} =
  \frac{L N_\text{HC}}{\xi^o}, \\
  V^g & = & \frac{N_o^g + N_g^g}{\xi^g} =
  \frac{(1-L) N_\text{HC}}{\xi^g}.
\end{eqnarray}
Pressure derivatives are
\begin{eqnarray}
  \frac{\partial V^o}{\partial p} & = &
  \frac{V^o}{L} \frac{\partial L}{\partial p} -
  \frac{V^o}{\xi^o} \left(
    \frac{\partial \xi^o}{\partial p} +
    \frac{\partial \xi^o}{\partial C_g^o}
    \frac{\partial C_{g,B}^o}{\partial p}
  \right), \\
  \frac{\partial V^g}{\partial p} & = &
  -\frac{V^g}{1-L} \frac{\partial L}{\partial p} -
  \frac{V^g}{\xi^g} \left(
    \frac{\partial \xi^g}{\partial p} +
    \frac{\partial \xi^g}{\partial C_o^g}
    \frac{\partial C_{o,D}^g}{\partial p}
  \right),
\end{eqnarray}
temperature derivatives are
\begin{eqnarray}
  \frac{\partial V^o}{\partial T} & = &
  \frac{V^o}{L} \frac{\partial L}{\partial T} -
  \frac{V^o}{\xi^o} \left(
    \frac{\partial \xi^o}{\partial T} +
    \frac{\partial \xi^o}{\partial C_g^o}
    \frac{\partial C_{g,B}^o}{\partial T}
  \right), \\
  \frac{\partial V^g}{\partial T} & = &
  -\frac{V^g}{1-L} \frac{\partial L}{\partial T} -
  \frac{V^g}{\xi^g} \left(
    \frac{\partial \xi^g}{\partial T} +
    \frac{\partial \xi^g}{\partial C_o^g}
    \frac{\partial C_{o,D}^g}{\partial T}
  \right),
\end{eqnarray}
and molar mass derivatives are ($\nu=o,g$)
\begin{eqnarray}
  \frac{\partial V^o}{\partial N_\nu} & = &
  \frac{1}{\xi^o}\left(
    \frac{\partial L}{\partial N_\nu} N_\text{HC} +
    L \right), \\
  \frac{\partial V^g}{\partial N_\nu} & = &
  \frac{1}{\xi^g}\left(
    -\frac{\partial L}{\partial N_\nu} N_\text{HC} +
    (1-L) \right),
\end{eqnarray}
It remains to differentiate the liquid hydrocarbon fraction $L$:
\begin{eqnarray}
  \frac{\partial L}{\partial p} & = &
  \frac{1}{C_{o,D}^g - \left(1-C_{g,B}^o\right)}
  \left(
    (1-L) \frac{\partial C_{o,D}^g}{\partial p} -
    L \frac{\partial C_{g,B}^o}{\partial p}
  \right), \\
  \frac{\partial L}{\partial T} & = &
  \frac{1}{C_{o,D}^g - \left(1-C_{g,B}^o\right)}
  \left(
    (1-L) \frac{\partial C_{o,D}^g}{\partial T} -
    L \frac{\partial C_{g,B}^o}{\partial T}
  \right), \\
  \frac{\partial L}{\partial N_o} & = &
  \frac{1}{C_{o,D}^g - \left(1-C_{g,B}^o\right)}
  \frac{Z_o-1}{N_\text{HC}}, \\
  \frac{\partial L}{\partial N_g} & = &
  \frac{1}{C_{o,D}^g - \left(1-C_{g,B}^o\right)}
  \frac{Z_o}{N_\text{HC}}.
\end{eqnarray}

%----------------------------------------------------------------------

\csubsection{Data verification}

As the black-oil phase split is determined by user-supplied tables, it
is quite possible for an unphysical situation to arise. This may be
avoided by a set of tests.

%......................................................................

\subsubsection{Oil component distribution}

It is reasonable to assume that there is more of the heavy oil
component in the oil phase than in the gas phase. Thus:
\begin{equation}
  C_{o,D}^g < C_{o,B}^o = 1 - C_{g,B}^o.
\end{equation}
This also ensures that the denominator in the expressions for the
derivatives of $L$ is negative.

%......................................................................

\subsubsection{Oil/gas mass densities}

A black-oil fluid must give mass densities satisfying
\begin{equation}
  \label{eqn:rhoorhog}
  \rho^o \geq \rho^g.
\end{equation}
The binary system yields
\begin{equation}
  \rho^\ell = \frac{M_o N_o^\ell + M_g N_g^\ell}{V^\ell}
  = \left(M_o C_o^\ell + M_g C_g^\ell\right) \xi^\ell
\end{equation}
Then \eqref{eqn:rhoorhog} becomes
\begin{equation}
  \left[M_o \left(1-C_g^o\right) + M_g C_g^o\right] \xi^o \geq
  \left[M_o C_o^g + M_g \left(1-C_o^g\right)\right] \xi^g.
\end{equation}
The pseudo-component molecular weights are
\begin{eqnarray}
  M_o & = & \frac{\sum_{\nu\in\mathcal{O}} M_\nu C_\nu}{\sum_{\nu\in\mathcal{O}} C_\nu}, \\
  M_g & = & \frac{\sum_{\nu\in\mathcal{G}} M_\nu C_\nu}{\sum_{\nu\in\mathcal{G}} C_\nu}.
\end{eqnarray}
Failure of this check may be because the gas molar density is too
large relative to the oil molar density.

%......................................................................

\subsubsection{Compressibility check}

Increasing the pressure should decrease the total hydrocarbon
volumes. Thus
\begin{equation}
  \frac{\partial\left(V^o + V^g\right)}{\partial p} \leq 0.
\end{equation}
If the bubble- and dew-point curves are constant, then this reduces to
\begin{equation}
  \frac{V^o}{\xi^o} \frac{\partial\xi^o}{\partial p} +
  \frac{V^g}{\xi^g} \frac{\partial\xi^g}{\partial p} \geq 0.
\end{equation}

%......................................................................

\subsubsection{Mass/volume check}

Adding hydrocarbons should increase the total hydrocarbon volumes:
\begin{equation}
  \frac{\partial\left(V^o + V^g\right)}{\partial N_\nu} \geq 0,
  \quad\nu=o,g.
\end{equation}

%----------------------------------------------------------------------

\csubsection{Inverse grouping}

Define the grouped fractions as
\begin{eqnarray}
  C_g & = & \sum_{\nu\in\mathcal{G}} C_\nu, \\
  C_o & = & \sum_{\nu\in\mathcal{O}} C_\nu.
\end{eqnarray}
Then the oil phase composition is
\begin{equation}
  C_\nu^o = C_\nu \left\{
    \begin{array}{ll}
      C_g^o / C_g, & \nu\in\mathcal{G}, \\
      C_o^o / C_o, & \nu\in\mathcal{O},
    \end{array}
  \right.
\end{equation}
and the gas phase composition is
\begin{equation}
  C_\nu^g = C_\nu \left\{
    \begin{array}{ll}
      C_g^g / C_g, & \nu\in\mathcal{G}, \\
      C_o^g / C_o, & \nu\in\mathcal{O}.
    \end{array}
  \right.
\end{equation}
The grouped phase compositions, $C_\nu^\ell$, are known from the phase
split.

For the volume derivatives with moles, the inverse grouping is
approximated as follows:
\begin{equation}
  \frac{\partial V^\ell}{\partial N_\nu} = \left\{
    \begin{array}{ll}
      \partial V^\ell / \partial N_g, & \nu\in\mathcal{G}, \\
      \partial V^\ell / \partial N_o, & \nu\in\mathcal{O},
    \end{array}
  \right.
\end{equation}

%----------------------------------------------------------------------

\csubsection{Viscosity and enthalpy}

The oil and gas viscosities are the given functions:
\begin{eqnarray}
  \mu^o & = & \mu^o\left(p,C_g^o,T\right), \\
  \mu^g & = & \mu^g\left(p,C_o^g,T\right).
\end{eqnarray}
Likewise, the oil gas molar enthalpies are these given functions:
\begin{eqnarray}
  \bar h^o & = & \bar h^o\left(p,C_g^o,T\right), \\
  \bar h^g & = & \bar h^g\left(p,C_o^g,T\right).
\end{eqnarray}
The energy conservation law requires the enthalpy density $h^\ell
\rho^\ell$ and its temperature derivative:
\begin{eqnarray}
  h^\ell \rho^\ell & = & \bar h^\ell \xi^\ell, \\
  \frac{\partial\left(h^o \rho^o\right)}{\partial T} & = &
  \left(
    \frac{\partial \bar h^o}{\partial T} +
    \frac{\partial \bar h^o}{\partial C_g^o}
    \frac{\partial C_g^o}{\partial T}
  \right) \xi^o +
  \bar h^o \left(
    \frac{\partial \xi^o}{\partial T} +
    \frac{\partial \xi^o}{\partial C_g^o}
    \frac{\partial C_g^o}{\partial T}
  \right), \\
  \frac{\partial\left(h^g \rho^g\right)}{\partial T} & = &
  \left(
    \frac{\partial \bar h^g}{\partial T} +
    \frac{\partial \bar h^g}{\partial C_o^g}
    \frac{\partial C_o^g}{\partial T}
  \right) \xi^g +
  \bar h^g \left(
    \frac{\partial \xi^g}{\partial T} +
    \frac{\partial \xi^g}{\partial C_o^g}
    \frac{\partial C_o^g}{\partial T}
  \right).
\end{eqnarray}
The derivatives of $C_g^o$ and $C_o^g$ are zero in the single-phase
region, otherwise the derivatives of the bubble- and dew point curves
are used.

%======================================================================

\csection{Cubic equation of state}

A cubic equation of state uses the relation for a real gas
\begin{equation}
  p V^\ell = Z^\ell N^\ell R T,
\end{equation}
where $R=8.3145\ \joule\per(\mole\usk\kelvin)$ is the universal gas
constant, and $Z^\ell$ is the phase compressibility factor.

The cubic equation for the phase compressibility may be expressed
\begin{equation}
  (Z^\ell)^3 + a_2^\ell (Z^\ell)^2 + a_1^\ell (Z^\ell) + a_0^\ell = 0,
\end{equation}
in which
\begin{eqnarray}
  a_2^\ell & = & -1 - B^\ell (1+\delta_1+\delta_2), \\
  a_1^\ell & = & A^\ell + B^\ell
  \left( 1 + B^\ell \right) \left(\delta_1 + \delta_2\right) +
  \delta_1\delta_2 \left(B^\ell\right)^2, \\
  a_0^\ell & = & - A^\ell B^\ell - \left(B^\ell\right)^2
  \left( 1 + B^\ell \right)\delta_1\delta_2.
\end{eqnarray}
Two common choices of the parameters herein are:
\begin{description}
\item[Soave-Redlich-Kwong] $\delta_1=-1$ and $\delta_2=0$, with
  $\Omega_a=0.42748$ and $\Omega_b=0.08664$, and
  \begin{equation}
    m(\omega_i) = 0.48508 + 1.55171 \omega_i - 0.15613 \omega_i^2.
  \end{equation}
\item[Peng-Robinson] $\delta_1=-(1+\sqrt{2})$ and
  $\delta_2=-(1-\sqrt{2})$, with $\Omega_a=0.45724$ and
  $\Omega_b=0.0778$, and
  \begin{equation}
    m(\omega_i) = \left\{
      \begin{array}{ll}
        0.37464 + 1.53226 \omega_i - 0.2699 \omega_i^2, & \omega_i
        \leq 0.49, \\
        0.379642 + 1.48503 \omega_i - 0.164423 \omega_i^2 + 0.016666
        \omega_i^3, & \omega_i > 0.49.
      \end{array}
    \right.
  \end{equation}
\end{description}
$\omega_i$ is the acentric factor of component $i$.

%----------------------------------------------------------------------

\csubsection{Phase/component parameters and derivatives}

The following lists component/phase parameters and derivatives which
will be used in the phase splitting (flash) algorithms and subsequent
phase property determination.

%......................................................................

\subsubsection{Component parameters}

Pure component parameters are:
\begin{eqnarray}
  a_i & = & \frac{\Omega_a R^2 T_{i,c}^2}{p_{i,c}} \alpha_i^2, \\
  b_i & = & \frac{\Omega_b  R T_{i,c}}{p_{i,c}}, \\
  \alpha_i & = & 1 + m(\omega_i) \left(
    1 - \sqrt{\frac{T}{T_{i,c}}}\right).
\end{eqnarray}
in which $T_{i,c}$ is the critical component temperature and $p_{i,c}$
is the critical component pressure, and
\begin{eqnarray}
  \frac{\partial a_i}{\partial T} & = &
  2\frac{a_i}{\alpha_i} \frac{\partial\alpha_i}{\partial T}, \\
  \frac{\partial\alpha_i}{\partial T} & = &
  -\frac{m(\omega_i)}{2}\frac{1}{\sqrt{T_{i,c} T}}.
\end{eqnarray}
It will also be useful to define
\begin{equation}
  a_{ij} = \left( 1 - d_{ij}\right) \sqrt{a_i a_j},
\end{equation}
where $d_{ij}$ are the binary interaction parameters, satisfying
$d_{ij}=d_{ji}$. Its temperature derivative is
\begin{equation}
  \frac{\partial a_{ij}}{\partial T} =
  \left(
    1 - d_{ij}\right) \frac{a_j \frac{\partial a_i}{\partial T} +
    a_i \frac{\partial a_j}{\partial T}}{2\sqrt{a_i a_j}}.
\end{equation}

%......................................................................

\subsubsection{Phase parameters}

Phase parameters and derivatives are:
\begin{eqnarray}
  a^\ell & = & \sum_i \sum_j C_i^\ell C_j^\ell a_{ij}, \\
  b^\ell & = & \sum_i C_i^\ell b_i, \\
  \frac{\partial a^\ell}{\partial N_i^\ell} & = &
  \frac{2}{N^\ell} \left(
    \sum_j C_j^\ell a_{ij} - a^\ell \right), \\
  \frac{\partial b^\ell}{\partial N_i^\ell} & = &
  \frac{b_i - b^\ell}{N^\ell}, \\
  \frac{\partial a^\ell}{\partial T} & = &
  \sum_i \sum_j C_i^\ell C_j^\ell \frac{\partial a_{ij}}{\partial T}.
\end{eqnarray}
Then
\begin{eqnarray}
  A^\ell & = & \frac{p a^\ell}{(RT)^2}, \\
  B^\ell & = & \frac{p b^\ell}{RT}.
\end{eqnarray}
Pressure derivatives are
\begin{eqnarray}
  \frac{\partial A^\ell}{\partial p} & = & \frac{A^\ell}{p}, \\
  \frac{\partial B^\ell}{\partial p} & = & \frac{B^\ell}{p},
\end{eqnarray}
temperature derivatives are
\begin{eqnarray}
  \frac{\partial A^\ell}{\partial T} & = & -2\frac{A^\ell}{T} +
  \frac{A^\ell}{a^\ell} \frac{\partial a^\ell}{\partial T}, \\
  \frac{\partial B^\ell}{\partial T} & = & -\frac{B^\ell}{T},
\end{eqnarray}
and mole derivatives are
\begin{eqnarray}
  \frac{\partial A^\ell}{\partial N_i^\ell} & = &
  \frac{A^\ell}{a^\ell} \frac{\partial a^\ell}{\partial N_i^\ell}, \\
  \frac{\partial B^\ell}{\partial N_i^\ell} & = &
  \frac{B^\ell}{b^\ell} \frac{\partial b^\ell}{\partial N_i^\ell}.
\end{eqnarray}

%......................................................................

\subsubsection{Phase compressibility}

The roots of the cubic equation may be found by letting
\begin{eqnarray}
  Q & = & \frac{3 a_0 - a_2^2}{9}, \\
  R & = & \frac{9 a_2 a_1 - 27 a_0 - 2 a_2^2}{54}, \\
  D & = & Q^3 + R^2.
\end{eqnarray}
If $D>0$ or $Q=0$, there is just one real root, given by
\begin{eqnarray}
  S & = & \sqrt[3]{R + \sqrt{D}}, \\
  T & = & \sqrt[3]{R - \sqrt{D}}, \\
  Z & = & -\frac{a_2}{3} + S + T.
\end{eqnarray}
Otherwise, there are three real roots, which are given by
\begin{eqnarray}
  \theta & = & \cos^{-1}\left(\frac{R}{\sqrt{-Q^3}}\right), \\
  Z_1 & = & 2\sqrt{-Q}\cos\left(\frac{\theta}{3}\right) - \frac{a_2}{3}, \\
  Z_2 & = & 2\sqrt{-Q}\cos\left(\frac{\theta+2\pi}{3}\right) - \frac{a_2}{3}, \\
  Z_3 & = & 2\sqrt{-Q}\cos\left(\frac{\theta+4\pi}{3}\right) - \frac{a_2}{3}.
\end{eqnarray}
The largest root is associated with the gas phase, and the smallest to
the oil phase.

Derivatives of $Z^\ell$ are given by differentiation through the cubic
equation, yielding
\begin{equation}
  \frac{\partial Z^\ell}{\partial X} = -\frac{
    \frac{\partial a_2^\ell}{\partial X} Z^2 +
    \frac{\partial a_1^\ell}{\partial X} Z +
    \frac{\partial a_0^\ell}{\partial X}}{
    3 Z^2 + 2 a_2^\ell Z + a_1^\ell}.
\end{equation}
Derivatives herein are
\begin{eqnarray}
  \frac{\partial a_2^\ell}{\partial X} & = &
  -\frac{\partial B^\ell}{\partial X}
  (1+\delta_2+\delta_2), \\
  \frac{\partial a_1^\ell}{\partial X} & = &
  \frac{\partial A^\ell}{\partial X} +
  \frac{\partial B^\ell}{\partial X} \left[
    \left( 1 + 2 B^\ell\right)
    \left( \delta_1 + \delta_2 \right) +
    2\delta_1\delta_2 B^\ell \right], \\
  \frac{\partial a_0^\ell}{\partial X} & = &
  -\frac{\partial A^\ell}{\partial X} B^\ell -
  \frac{\partial B^\ell}{\partial X} \left[
    A^\ell + B^\ell \left[ 2 + 3 B^\ell \right]
    \delta_1 \delta_2 \right],
\end{eqnarray}
where $X$ may be $p, T$, or $N_i^\ell$.

%......................................................................

\subsubsection{Volume translation}

For improved liquid property predictions, volume translations may be
performed. Then the phase compressibilities and derivatives become
\begin{eqnarray}
  Z^\ell & \leftarrow & Z^\ell - \frac{p}{RT} \sum_i C_i^\ell c_i, \\
  \frac{\partial Z^\ell}{\partial p} & \leftarrow &
  \frac{\partial Z^\ell}{\partial p} - \frac{1}{RT}
  \sum_i C_i^\ell c_i, \\
  \frac{\partial Z^\ell}{\partial T} & \leftarrow &
  \frac{\partial Z^\ell}{\partial T} + \frac{p}{RT^2}
  \sum_i C_i^\ell c_i, \\
  \frac{\partial Z^\ell}{\partial N_i^\ell} & \leftarrow &
  \frac{\partial Z^\ell}{\partial N_i^\ell} -
  \frac{p}{RT} \frac{c_i - \sum_j C_j^\ell c_j}{N^\ell}.
\end{eqnarray}
$c_i$ is a component shift parameter, and may be related to the
dimensionless shift parameter $s_i$ by
\begin{equation}
  c_i = s_i b_i.
\end{equation}

%......................................................................

\subsubsection{Component fugacities}

Fugacities for each component in each phase is given by
\begin{eqnarray}
  f_i^\ell & = & p C_i^\ell \varphi_i^\ell, \\
  \varphi_i^\ell & = & y_i^\ell\left(w^\ell\right)^{v_i^\ell}, \\
  y_i^\ell & = & \frac{\exp\left(\beta_i^\ell (Z^\ell-1)\right)}
  {Z^\ell - B^\ell}, \\
  w^\ell & = & \frac{Z^\ell - \delta_2 B^\ell}{Z^\ell - \delta_1
    B^\ell}, \\
  v_i^\ell & = & \frac{A^\ell\left(\alpha_i^\ell-\beta_i^\ell\right)}
  {(\delta_2-\delta_1)B^\ell}, \\
  \alpha_i^\ell & = & \frac{2}{a^\ell}\sum_j C_j^\ell a_{ij}, \\
  \beta_i^\ell & = & \frac{b_i}{b^\ell}.
\end{eqnarray}
$\varphi_i^\ell$ is the fugacity coefficient. Fugacity derivatives are
\begin{eqnarray}
  \frac{\partial f_i^\ell}{\partial p} & = &
  \frac{f_i^\ell}{p} + \frac{f_i^\ell}{\varphi_i^\ell}
  \frac{\partial\varphi_i^\ell}{\partial p}, \\
  \frac{\partial f_i^\ell}{\partial T} & = &
  \frac{f_i^\ell}{\varphi_i^\ell}
  \frac{\partial\varphi_i^\ell}{\partial T}, \\
  \frac{\partial f_i^\ell}{\partial N_j^\ell} & = &
  \frac{f_i^\ell}{C_i^\ell} \frac{\partial C_i^\ell}{\partial N_j^\ell} +
  \frac{f_i^\ell}{\varphi_i^\ell}\frac{\partial\varphi_i^\ell}{\partial N_j^\ell}, \\
  \frac{\partial C_i^\ell}{\partial N_j^\ell} & = &
  \frac{\delta_{ij} - C_i^\ell}{N^\ell}.
\end{eqnarray}
Derivatives of the fugacity coefficient is
\begin{eqnarray}
  \frac{\partial\varphi_i^\ell}{\partial X} & = &
  \frac{\partial y_i^\ell}{\partial X} \frac{\varphi_i^\ell}{y_i^\ell} +
  \varphi_i^\ell\left(
    \frac{\partial v_i^\ell}{\partial X} \ln w^\ell +
    \frac{v_i^\ell}{w^\ell}\frac{\partial w^\ell}{\partial X}
  \right), \\
  \frac{\partial y_i^\ell}{\partial X} & = &
  \left(
    \frac{\partial\beta_i^\ell}{\partial X} (Z^\ell-1) +
    \beta_i^\ell \frac{\partial Z^\ell}{\partial X}
  \right) y_i^\ell -
  \frac{y_i^\ell}{Z^\ell-B^\ell}\left(
    \frac{\partial Z^\ell}{\partial X} -
    \frac{\partial B^\ell}{\partial X}
  \right), \\
  \frac{\partial v_i^\ell}{\partial X} & = &
  \frac{\frac{\partial A^\ell}{\partial
      X}\left(\alpha_i^\ell-\beta_i^\ell\right) +
    A^\ell\left(
      \frac{\partial\alpha_i^\ell}{\partial X} -
      \frac{\partial\beta_i^\ell}{\partial X}
    \right)}{\left(\delta_2-\delta_1\right) B^\ell} -
  \frac{v_i^\ell}{B^\ell}
  \frac{\partial B^\ell}{\partial X}, \\
  \frac{\partial w^\ell}{\partial X} & = &
  \frac{\frac{\partial Z^\ell}{\partial X} -
    \delta_2 \frac{\partial B^\ell}{\partial X}
    - w^\ell \left(
      \frac{\partial Z^\ell}{\partial X} -
      \delta_1 \frac{\partial B^\ell}{\partial X}
    \right)}{Z^\ell - \delta_1 B^\ell}.
\end{eqnarray}
As before, $X$ may denote pressure, temperature, or a mole number.
Derivatives of $\alpha_i^\ell$ and $\beta_i^\ell$ are
\begin{eqnarray}
  \frac{\partial\alpha_i^\ell}{\partial T} & = &
  -\frac{\alpha_i^\ell}{a^\ell}\frac{\partial a^\ell}{\partial T} +
  \frac{2}{a^\ell} \sum_j C_j^\ell \frac{\partial a_{ij}}{\partial T}, \\
  \frac{\partial\alpha_i^\ell}{\partial C_j^\ell} & = &
  -\frac{\alpha_i^\ell}{a^\ell}\frac{\partial a^\ell}{\partial C_j^\ell} +
  \frac{2}{a^\ell} a_{ij}, \\
  \frac{\partial\beta_i^\ell}{\partial N_j^\ell} & = &
  -\frac{\beta_i^\ell}{b^\ell}\frac{\partial b^\ell}{\partial N_j^\ell}.
\end{eqnarray}

%----------------------------------------------------------------------

\csubsection{Flash algorithms}

When splitting a composition into oil and gas, overall material
balance must be maintained. First define
\begin{equation}
  K_i = \frac{C_i^g}{C_i^o},
\end{equation}
and
\begin{equation}
  C_i^\text{HC} = \frac{N_i}{N_\text{HC}} = L C_i^o + V C_i^g =
  L C_i^o + (1-L) C_i^g,
\end{equation}
where $L=N^o/N_\text{HC}$ is the liquid hydrocarbon fraction and
$V=N^g/N_\text{HC}$ is the vapor hydrocarbon fraction. It follows
that
\begin{eqnarray}
  C_i^o & = & \frac{C_i^\text{HC}}{L+(1-L) K_i}, \label{eqn:cio} \\
  C_i^g & = & K_i C_i^o = \frac{K_i C_i^\text{HC}}{L+(1-L) K_i}. \label{eqn:cig}
\end{eqnarray}
The liquid fraction yielding material balance is solved from
\begin{equation}
  0 = \sum_i \left( C_i^g - C_i^o \right) =
  \sum_i \frac{(K_i-1) C_i^\text{HC}}{L+(1-L) K_i}
  = f(L).
\end{equation}
$f(L)=0$ is the Rachford-Rice equation, and it may be solved by
Newton's method. Its derivative is
\begin{equation}
  f'(L) = \sum_i\frac{(1-K_i)^2 C_i^\text{HC}}
  {\left(L+(1-L) K_i\right)^2}.
\end{equation}
If $L<0$, then there is no free oil, and if $L>1$, there will not be
any free gas.

%......................................................................

\subsubsection{Successive substitution}

The successive substitution uses the calculated fugacities, and it does
achieve a thermodynamical equilibrium. It goes as follows:
\begin{enumerate}
\item Solve the Rachford-Rice equation for $L$.
\item Determine $C_i^o$ and $C_i^g$ from
  Equation~\eqref{eqn:cio}-\eqref{eqn:cig}.
\item Calculate new fugacities, which includes solving the cubic
  equation of state.
\item Determine new $K_i$-values from
  \begin{equation}
    K_i = \frac{\varphi_i^o}{\varphi_i^g}.
  \end{equation}
\end{enumerate}
This iteration is continued until the fugacities do become equal one
another. Of course, should the iteration enter a single-phase region,
it should be stopped.

Initial estimates on the $K_i$ values can be determined from Wilson's
correlation formula:
\begin{equation}
  K_i = \frac{p_{i,c}}{p}\exp\left[
    5.373 (1+\omega_i) \left(
      1-\frac{T_{i,c}}{T}
    \right)
  \right].
\end{equation}

%......................................................................

\subsubsection{Newton-Raphson iteration}

The fugacity equality can be solved directly using Newton's
method. Set
\begin{equation}
  r_i = f_i^o - f_i^g.
\end{equation}
Since $N_i^g=N_i-N_i^o$, it suffices to solve for the oil phase
composition. The residual derivative is
\begin{equation}
  \frac{\partial r_i}{\partial N_j^o} =
  \frac{\partial f_i^o}{\partial N_j^o} -
  \frac{\partial f_i^g}{\partial N_j^g}
  \frac{\partial N_j^g}{\partial N_j^o} =
  \frac{\partial f_i^o}{\partial N_j^o} +
  \frac{\partial f_i^g}{\partial N_j^g}.
\end{equation}
This iteration gives a system of equations to be solved at each step.
The molar mass update may yield a negative number of moles for certain
components, in which case a scaling of the update must be done.

%......................................................................

\subsubsection{Overall flash procedure}

An overall procedure is to apply successive substitution for a few
iterations, then switch to the Newton-Raphson method. This switching
may be done by monitoring the residual $f_i^o-f_i^g$, and using
Newton's method for small residuals, and the successive substitution
for larger values.

%----------------------------------------------------------------------

\csubsection{Phase properties and derivatives}

Volumes and densities are obtained from the equation of state:
\begin{eqnarray}
  V^\ell & = & \frac{Z^\ell N^\ell RT}{p}, \\
  \xi^\ell & = & \frac{p}{Z^\ell RT}.
\end{eqnarray}
Volume derivatives are
\begin{eqnarray}
  \frac{\partial V^\ell}{\partial p} & = &
  \frac{V^\ell}{Z^\ell} \left(
    \frac{\partial Z^\ell}{\partial p} +
    \sum_i
    \frac{\partial Z^\ell}{\partial N_i^\ell}
    \frac{\partial N_i^\ell}{\partial p}
  \right) +
  \frac{V^\ell}{N^\ell} \frac{\partial N^\ell}{\partial p} -
  \frac{V^\ell}{p}, \\
  \frac{\partial V^\ell}{\partial T} & = &
  \frac{V^\ell}{Z^\ell} \left(
    \frac{\partial Z^\ell}{\partial T} +
    \sum_i
    \frac{\partial Z^\ell}{\partial N_i^\ell}
    \frac{\partial N_i^\ell}{\partial T}
  \right) +
  \frac{V^\ell}{N^\ell} \frac{\partial N^\ell}{\partial T} +
  \frac{V^\ell}{T}, \\
  \frac{\partial V^\ell}{\partial N_i} & = &
  \sum_j \left(
    \frac{V^\ell}{Z^\ell}
    \frac{\partial Z^\ell}{\partial N_j^\ell} +
    \frac{V^\ell}{N^\ell}
  \right) \frac{\partial N_j^\ell}{\partial N_i}.
\end{eqnarray}
Mass derivatives are determined differently depending on the number of
phases. For a single hydrocarbon phase;
\begin{eqnarray}
  \frac{\partial N^\ell}{\partial p} & = & 0, \\
  \frac{\partial N^\ell}{\partial T} & = & 0, \\
  \frac{\partial N_i^\ell}{\partial N_j} & = & \delta_{ij}.
\end{eqnarray}
Here, $\ell$ is the present phase. If both oil and gas are present,
the following relations reduce the computations somewhat. Starting
with
\begin{equation}
  \frac{\partial N_i}{\partial X} =
  \frac{\partial N_i^o}{\partial X} +
  \frac{\partial N_i^g}{\partial X},
\end{equation}
it follows that for $X=p$ or $T$, the left hand side is zero, since
$N_i$ is an independent primary variable. Then
\begin{eqnarray}
  \frac{\partial N_i^g}{\partial p} & = &
  -\frac{\partial N_i^o}{\partial p}, \\
  \frac{\partial N_i^g}{\partial T} & = &
  -\frac{\partial N_i^o}{\partial T}.
\end{eqnarray}
If $X=N_j$, this becomes
\begin{equation}
  \frac{\partial N_i^g}{\partial N_j} =
  \delta_{ij} - \frac{\partial N_i^o}{\partial N_j}.
\end{equation}

The determination of $\partial N_i^\ell/\partial X$ for a two-phase
case will be done using the fugacity equality condition. The total
derivative of the component fugacity is
\begin{equation}
  \frac{d f_i^\ell}{d X} =
  \frac{\partial f_i^\ell}{\partial p}
  \frac{\partial p}{\partial X} +
  \frac{\partial f_i^\ell}{\partial T}
  \frac{\partial T}{\partial X} +
  \sum_j
  \frac{\partial f_i^\ell}{\partial N_j^\ell}
  \frac{\partial N_j^\ell}{\partial X}.
\end{equation}
As $f_i^o - f_i^g = 0$, so must their total derivatives. Hence
\begin{equation}
  \left(
    \frac{\partial f_i^o}{\partial p} -
    \frac{\partial f_i^g}{\partial p}
  \right)
  \frac{\partial p}{\partial X} +
  \left(
    \frac{\partial f_i^o}{\partial T} -
    \frac{\partial f_i^g}{\partial T}
  \right)
  \frac{\partial T}{\partial X} +
  \sum_j
  \left(
    \frac{\partial f_i^o}{\partial N_j^o}
    \frac{\partial N_j^o}{\partial X} -
    \frac{\partial f_i^g}{\partial N_j^g}
    \left[
      \frac{\partial N_j}{\partial X} -
      \frac{\partial N_j^o}{\partial X}
    \right]
  \right) = 0.
\end{equation}
This forms a linear system of equations in the unknowns $\partial
N_j^o/\partial X$. Matrix entries are
\begin{equation}
  F_{ij} = 
  \frac{\partial f_i^o}{\partial N_j^o} +
  \frac{\partial f_i^g}{\partial N_j^g},
\end{equation}
and the right hand side entries are
\begin{equation}
  r_i =
  \left(
    \frac{\partial f_i^g}{\partial p} -
    \frac{\partial f_i^o}{\partial p}
  \right)
  \frac{\partial p}{\partial X} +
  \left(
    \frac{\partial f_i^g}{\partial T} -
    \frac{\partial f_i^o}{\partial T}
  \right)
  \frac{\partial T}{\partial X} +
  \sum_j
  \frac{\partial f_i^g}{\partial N_j^g}
  \frac{\partial N_j}{\partial X}.
\end{equation}
Notice that the system matrix is independent of $X$, hence it can be
factored just once, and then used for solving for pressure,
temperature and molar mass derivatives.

%----------------------------------------------------------------------

\csubsection{Viscosity}

Viscosity of oil and gas may be calculated from the Lohrenz-Bray-Clark
(LBC) formula. It proceeds by first calculating mole fraction averaged
component properties:
\begin{eqnarray}
  V_c^\ell & = & \sum_i C_i^\ell V_{i,c}, \\
  T_c^\ell & = & \sum_i C_i^\ell T_{i,c}, \\
  M^\ell & = & \sum_i C_i^\ell M_i, \\
  p_c^\ell & = & \sum_i C_i^\ell p_{i,c}, \\
  M_2^\ell & = & \sum_i C_i^\ell \sqrt{M_i}, \\
  \mu_*^\ell & = & \frac{\sum_i C_i^\ell \mu_i \sqrt{M_i}}{M_2^\ell}.
\end{eqnarray}
$V_{i,c}$ is the critical component molar volume. The diluted gas
viscosity for each component is given by
\begin{equation}
  \mu_i = \zeta_i^{-1} \left\{
    \begin{array}{ll}
      c_1 T_{i,r}^{e_1}, & T_{i,r} < T_\text{div}, \\
      c_2 \left( c_3 T_{i,r} - c_4 \right)^{e_2}, &
      T_{i,r} \geq T_\text{div}.
    \end{array}
  \right.
\end{equation}
The viscosity reducing parameter $\zeta_i$ is
\begin{equation}
  \zeta_i^{-1} = \frac{M_i^{1/2} p_{i,c}^{2/3}}{T_{i,c}^{1/6}},
\end{equation}
the reduced temperature is
\begin{equation}
  T_{i,r} = \frac{T}{T_{i,c}},
\end{equation}
and the constants are
\begin{eqnarray}
  c_1 & = & 3.4\cdot 10^{-4}, \\
  c_2 & = & 1.778\cdot 10^{-4}, \\
  c_3 & = & 4.58, \\
  c_4 & = & 1.67, \\
  e_1 & = & 0.94, \\
  e_2 & = & 0.625, \\
  T_\text{div} & = & 1.5.
\end{eqnarray}

The phase viscosity is then
\begin{equation}
  \mu^\ell = \mu_*^\ell + \left(\zeta^\ell\right)^{-1}
  \left( (\chi^\ell)^4 - 10^{-4} \right),
\end{equation}
where
\begin{eqnarray}
  \left(\zeta^\ell\right)^{-1} & = &
  \frac{\left(M^\ell\right)^{1/2} \left(p_c^\ell\right)^{2/3}}
  {\left(T_c^\ell\right)^{1/6}}, \\
  \chi^\ell & = &
  a_1 + \xi_r^\ell \left(
    a_2 + \xi_r^\ell \left(
      a_3 + \xi_r^\ell \left(
        a_4 + \xi_r^\ell a_5 \right)
    \right)
  \right), \\
  \xi_r^\ell & = & \xi^\ell V_c^\ell,
\end{eqnarray}
and the five new constants are
\begin{eqnarray}
  a_1 & = & 0.1023, \\
  a_2 & = & 0.023364, \\
  a_3 & = & 0.058533, \\
  a_4 & = & -0.040758, \\
  a_5 & = & 0.0093324.
\end{eqnarray}

%----------------------------------------------------------------------

\csubsection{Enthalpy}

The molar enthalpy of a hydrocarbon phase may be given by
\begin{equation}
  \bar h^\ell = \sum_i C_i^\ell M_i h_i^\ell.
\end{equation}
The component enthalpies in each phase are given through the heat
capacities
\begin{equation}
  c_i^\ell = c_{i,1}^\ell + c_{i,2}^\ell (T-T_0),
\end{equation}
where $T_0$ is a reference temperature, for instance at surface
conditions. Then
\begin{equation}
  h_i^\ell = \int_{T_0}^T c_i^\ell(T)\, dT =
  c_{i,0}^\ell + c_{i,1}^\ell (T-T_0) +
  \frac{c_{i,2}^\ell}{2} (T-T_0)^2.
\end{equation}
The heat capacities $c_{i,j}^\ell$ must be specified for each phase
and component. Unspecified values may default to zero.

The enthalpy density and heat capacity is given by
\begin{eqnarray}
  h^\ell \rho^\ell & = & \bar h^\ell \xi^\ell, \\
  \frac{\partial\left(h^\ell\rho^\ell\right)}{\partial T} & = &
  \frac{\partial\bar h^\ell}{\partial T} \xi^\ell +
  \bar h^\ell \frac{\partial\xi^\ell}{\partial T}, \\
  \frac{\partial\bar h^\ell}{\partial T} & = &
  \sum_i \left(
    \frac{\partial C_i^\ell}{\partial N_i^\ell}
    \frac{\partial N_i^\ell}{\partial T} M_i h_i^\ell +
    C_i^\ell M_i \frac{\partial h_i^\ell}{\partial T}
  \right), \\
  \frac{\partial C_i^\ell}{\partial N_i^\ell} & = &
  C_i^\ell \left(
    \frac{1}{N_i^\ell} - \frac{1}{N^\ell}
  \right), \\
  \frac{\partial h_i^\ell}{\partial T} & = & c_i^\ell, \\
  \frac{\partial\xi^\ell}{\partial T} & = &
  \frac{\xi^\ell}{N^\ell} \sum_i \frac{\partial N_i}{\partial T} -
  \frac{\xi^\ell}{V^\ell} \frac{\partial V^\ell}{\partial T}.
\end{eqnarray}

%======================================================================

\csection{Water properties}

The water phase properties are determined by the moles of component
$\nu=0$ and the functions
\begin{eqnarray}
  \xi^w & = & \xi^w\left(p,T\right), \\
  \bar h^w & = & \bar h^w\left(p,T\right), \\
  \mu^w & = & \mu^w\left(p,T\right).
\end{eqnarray}
Then its volume and associated properties are
\begin{equation}
  V^w = \frac{N^w}{\xi^w},\quad
  N^w = N_0^w = N_0,
\end{equation}
and the volume derivatives are
\begin{eqnarray}
  \frac{\partial V^w}{\partial p} & = &
  -\frac{V^w}{\xi^w} \frac{\partial\xi^w}{\partial p}, \\
  \frac{\partial V^w}{\partial T} & = &
  -\frac{V^w}{\xi^w} \frac{\partial\xi^w}{\partial T}, \\
  \frac{\partial V^w}{\partial N_w} & = &
  \frac{1}{\xi^w}.
\end{eqnarray}
Enthalpy density and the heat capacity are
\begin{eqnarray}
  h^w \rho^w & = & \bar h^w \xi^w, \\
  \frac{\partial\left(h^w \rho^w\right)}{\partial T} & = &
  \frac{\partial \bar h^w}{\partial T} \xi^w +
  \bar h^w \frac{\partial \xi^w}{\partial T}.
\end{eqnarray}

%%% Local Variables: 
%%% mode: latex
%%% TeX-master: "td"
%%% End: 
