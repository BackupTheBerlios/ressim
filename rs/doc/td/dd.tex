\chapter{Domain decomposition}

\minitoc

The solution of the linear equations is done in parallel using a
two-level overlapping domain decomposition method. First, let the grid
$U$ be split into the non-overlapping subgrids $U^i$. A small amount
of overlap between the subgrids is added, yielding $\hat U^i$.
Furthermore, $U^0$ is the coarse mesh with one cell for each subgrid
$U^i$.

$N_j^i$ is the set of local cellindices in $\hat U^i$ which overlap
with $U^j$. Hence $N_i^i$ are the local cellindices of $U^i$, and it
can be abbreviated as $N_i$. A piecewise constant restriction operator
$R_i$ taking a vector $x^i$ on $\hat U^i$ to cell $i$ in $U^0$ is
\begin{equation}
  R_i x^i = \sum_{I\in N_i^i} x_I^i.
\end{equation}
Likewise, the prolongation from cell $i$ in $U^0$ to $\hat U^i$ is
\begin{equation}
  P_i x^0 = x_i^0.
\end{equation}

%======================================================================

\csection{System setup}

On subdomain $i$, the local system
\begin{equation}
  A^i x^i = b^i
\end{equation}
is assembled. From this, row $i$ of the coarse system $A^0 x^0 = b^0$
is created as follows:
\begin{eqnarray}
  A_{ij}^0 & = & R_i A^i P_i = \sum_{I\in N_i^i} \sum_{J\in N_j^i} A_{IJ}^i, \\
  x_i^0 & = & R_i x^i = \sum_{I\in N_i^i} x_I^i, \\
  b_i^0 & = & R_i b^i = \sum_{I\in N_i^i} b_I^i.
\end{eqnarray}
This corresponds to piecewise constant interpolation. The two-level
iteration now proceeds:
\begin{enumerate}
\item Build the local matrices $A^i$ and vectors $b^i$, for all $i\geq
  1$.
\item Assemble the coarse matrix $A^0$ and vector $b^0$.
\item Initialize $x^i=0,i\geq 0$.
\item While $\|A^i x_{(k)}^i - b^i\|>\epsilon,\forall i\geq 1$, iterate:
  \begin{enumerate}
  \item Solve for $x_{(k+1)}^0$:
    \begin{equation}
      A^0 x_{(k+1)}^0 = b^0 - \sum_i R_i A^i x_{(k)}^i.
    \end{equation}
  \item Solve in parallel for $x_{(k+1)}^i$:
    \begin{equation}
      A^i x_{(k+1)}^i = b^i - A^i P_i x_{(k+1)}^0.
    \end{equation}
  \item Exchange information in the halo-region between the
    subdomains.
  \item Update $k\leftarrow k+1$.
  \end{enumerate}
\end{enumerate}

%%% Local Variables: 
%%% mode: latex
%%% TeX-master: "td"
%%% End: 
